% vim: spell
\documentclass[10pt]{sigplanconf}

% The following \documentclass options may be useful:
%
% 10pt  To set in 10-point type instead of 9-point.
% 11pt  To set in 11-point type instead of 9-point.
% authoryear To obtain author/year citation style instead of numeric.

\usepackage{amsmath}
\usepackage{bbding}
\usepackage{pifont}
\usepackage{url}

\begin{document}

\conferenceinfo{PLATEAU '12}{October 21, Tucson (AZ)} 
\copyrightyear{2012} 
\copyrightdata{[to be supplied]}

%\titlebanner{banner above paper title} % These are ignored unless
%\preprintfooter{short description of paper} % 'preprint' option specified.

\title{Programming Languages vs. Fat Fingers}
\subtitle{An Empirical Study}

\authorinfo{Diomidis Spinellis\and Vassilios Karakoidas}
  {Athens University of Economics and Business}
  {\{dds, bkarak\}@aueb.gr}

\maketitle

\begin{abstract}
We explore how programs written in ten popular programming languages
are affected by small changes of their source code.
This allows us to analyze the extend to which these languages
can detect simple errors at compile or at run time.
Our study is based on a large corpus of programs written in several programming
languages systematically perturbed using a mutation-based fuzz generator.
We found ... % XXX
\end{abstract}

\category{CR-number}{subcategory}{third-level}

\terms
term1, term2

\keywords
Programming Languages, Compiler, Fuzzing, Unit Testing

\section{Introduction} % {{{1
A substitution of a comma with a period in project Mercury's working
{\sc fortran} code compromised the accuracy of the results,
rendering them unsuitable for longer orbital missions \cite{Brad89,Neu95}.
How probable are such events and how does a programming language's
design affect their likelihood and severity?

To study these questions we chose ten popular programming languages,
and a corpus of programs written in all of them.
We then constructed a source code mutation {\em fuzzer}:
a tool that systematically introduces diverse random perturbations
into the program's source code,
and examined whether the resultant source code had errors that
were detected at compile or runtime, and whether it produced
erroneous results.

In practice,
the errors that we artificially introduced into the source code can
crop up in a number of ways.
Mistyping---the ``fat fingers'' syndrome-- is one plausible source.
Other scenarios include
absent-mindedness,
automated refactorings gone awry
(especially in languages where such tasks cannot be reliably implemented),
unintended consequences from complex editor commands or
search-and-replace operations,
and even the odd cat walking over the keyboard.

The contribution of our work is twofold.
First, we describe a method for systematically evaluating the tolerance
of source code written in diverse programming languages to a particular
class of errors.
In addition, we apply this method to numerous tasks written in ten popular
programming languages,
and by analyzing tens of thousands of cases we present an overview of
the likelihood and impact of these errors among diverse languages.

In the remainder of this paper we present our
methods (Section~\ref{sec:method}) and
results (Section~\ref{sec:results}),
we discuss our findings (Section~\ref{sec:discussion}),
compare our approach against related work
(Section~\ref{sec:related}),
and conclude with proposals for further work
(Section~\ref{sec:conclusions}).

\section{Methodology} % {{{1
\label{sec:method}

\begin{table}
\begin{center}
\begin{tabular}{ l l}
Language & Implementation \\
\hline
C 			& gcc 4.4.5 \\
C++ 		& g++ 4.4.5 \\
C\# 		& mono 2.6.7, CLI v2.0 \\
Haskell 	& ghc 6.12.1 \\
Java 		& OpenJDK 1.6.0\_18 \\
Javascript 	& spidermonkey 1.8.0 \\
PHP 		& PHP 5.3.3-7 \\
Perl 		& perl 5.10.1 \\
Python 		& python 2.6.6 \\
Ruby 		& ruby 1.5.8 \\
\end{tabular}
\end{center}
\caption{Tested languages.}
\label{tab:langs}
\end{table}

We selected the languages to test based on a number of sources
collated in an {\sc ieee} Spectrum article \cite{Kin11}:
an index created by
{\sc tiobe}\footnote{\url{http://www.tiobe.com/index.php/content/paperinfo/tpci/index.html}} (a software research firm),
the number of book titles listed on Powell's Books,
references in online discussions on {\sc irc}, and
number of job posts on Craigslist.
From the superset of the popular languages listed in those
sources we excluded
Actionscript, Visual Basic, {\sc sql}, Objective C, and the Unix shell,
because the corresponding programs or infrastructure would not match our methods.
According to the source composition,
our language coverage ranges from 71\% to 86\% of all languages.
The list of the ten languages we used in our study and the
particular implementations we used are listed in
Table~\ref{tab:langs}.

We obtained source code executing the same task in all ten
languages of our study from
{\em Rosetta Code},\footnote{\url{http://rosettacode.org/}}
a so-called programming chrestomathy site,
organized in the form of a wiki.
In the words of its creators,
the site aims is to present code for the same task in as many languages as possible,
thus demonstrating their similarities and differences and
aiding persons with a grounding in one approach to a problem in learning another.
At the time of the writing {\em Rosetta Code}
listed 600 tasks and code in 470 languages.
However, most of the tasks are presented only in a subset of those languages.

\begin{table}
\begin{center}
\begin{tabular}{ l p{5cm}}
Task Name & Description\\
\hline
AccumFactory & A function that takes a number $n$ and returns a function that takes a number $i$,
and returns $n$ incremented by $i$. \\
Beers & Print the ``99 bottles of beer on the wall'' song.\\
Dow & Detects all years in a range in which Christmas falls on a Sunday.\\
FlatList & Flattens a series of nested lists.\\
FuncComp & Implementation of mathematical function composition.\\
Horner & Horner's Method for polynomial evaluation.\\
Hello & A typical ``hello, world!'' program.\\
Mult & Ethiopian Multiplication: a method to multiply integers using only addition, doubling and halving.\\
MutRecursion & Hofstadter's Female and Male sequence~\cite{Hof89}.\\
ManBoy & A test to distinguish compilers that correctly implement
recursion and non-local references from those that do not~\cite{Knu64}.  \\
Power & Calculation of a set's $S$ power set: the set of all subsets of $S$.\\
Substring & Count the occurrences of a substring.\\
Tokenizer & A string tokenizing program.\\
ZigZag & Produce a square arrangement of the first $N^2$ integers,
where the numbers increase sequentially in a zig-zag along the anti-diagonals of the array.\\
\end{tabular}
\end{center}
\caption{List of the selected {\em Rosseta Code} tasks.}
\label{tab:tasks}
\end{table}

We selected our tasks from {\em Rosetta Code} through the following process.
First, we downloaded the listing of all available tasks and
filtered it to create a list of task name {\sc url}s.
We then downloaded the page for each task in MediWiki markup format,
located the headers for the languages in which that task was implemented, and
created a table containing tasks names and language names.
We joined that table with our chosen languages,
thus obtaining a count of the tasks implemented in
most of the languages in our set.
From that set we selected tasks that implemented diverse
non-trivial functionality,
and also, as a test case, the ``Hello, world!'' task.
The tasks we studied are listed in Table~\ref{tab:tasks}.

\begin{table*}
% ['c', 'cpp', 'cs', 'hs', 'java', 'js', 'php', 'pl', 'py', 'rb']
\begin{center}
\begin{tabular}{l r r r r r r r r r r   r}
 & C & C++ & C\# & Haskell & Java & Javascript & PHP & Perl & Python & Ruby & \textbf{Implemented}\\
 &   &     &     &         &      &            &     &      &        &      &  \textbf{Languages}\\
\hline
AccumFactory & 17 & 57 & 8 & 16 & 16 & 8 & 7 & 7 & 10 & 30 & 10 \\
Hello & 7 & 8 & 7 & 1 & 6 & 1 & 1 & 1 & 7 & 1 & 10 \\
FlatList & 118 & \ding{55} & \ding{55} & 15 & \ding{55} & 4 & 15 & 5 & 14 & 1 & 7 \\
Power & 27 & 77 & \ding{55} & 10 & 31 & 13 & 59 & 3 & 29 & 47 & 9 \\
ZigZag & 22 & 80 & \ding{55} & 19 & 46 & \ding{55} & 31 & 15 & 13 & 14 & 8 \\
FuncComp & 60 & 34 & 18 & 4 & 32 & 6 & 7 & 9 & 3 & 7 & 10 \\
Substring & 21 & 21 & 35 & \ding{55} & 10 & 1 & 3 & 9 & 1 & 1 & 9 \\
ManBoy & 46 & 32 & 22 & 11 & 28 & 8 & 13 & 8 & 11 & 5 & 10 \\
Beers & 14 & 12 & 28 & 6 & 21 & 9 & 14 & 20 & 13 & 12 & 10 \\
Tokenizer & 22 & 15 & 16 & \ding{55} & 11 & 1 & 3 & 1 & 2 & 1 & 9 \\
Horner & 21 & 20 & 15 & 3 & 22 & 3 & 8 & 10 & 6 & 3 & 10 \\
MutRecursion & 29 & 35 & 31 & 8 & 20 & 18 & 22 & 28 & 4 & 8 & 10 \\
Dow & 23 & 17 & 17 & 7 & 13 & 5 & 9 & 17 & 7 & 4 & 10 \\
Mult & 31 & 53 & 61 & 14 & 40 & 25 & 32 & 23 & 41 & 25 & 10 \\
\hline
\textbf{Total lines} & 458 & 461 & 258 & 114 & 296 & 102 & 224 & 156 & 161 & 159 & \\
\end{tabular}
\end{center}
\caption{Lines of Code per Task and per Language, Unimplemented Tasks, and Implemented Languages per Task.}
\label{tbl:lang-compatibility}
\end{table*}

Unfortunately, many of the tasks listed on {\em Rosetta Stone} were
not in a form that would allow us to execute them as part of our study.
Many would not compile, others lacked a test harness to produce output,
and some required specific installed libraries or particular
new language features.
We tried to fix as many problems as possible, but in the end the tasks
we ended up using were not as large or diverse as we would have liked.
In addition, we were unable to implement some of the tasks in all our
chosen languages.
Tasks written in Objective-C, which was initially part of our language set,
proved particularly tricky to compile,
mainly because we found it difficult to automate their compilation and
running.
Our impression was that the Objective-C tool chain are was
mostly geared toward a {\sc gui} environment.
Key size metrics of the tasks and languages we tested are listed in
Table~\ref{tbl:lang-compatibility}.

We implemented a language-agnostic fuzzer as a Perl script
that reads a program,
splits it into tokens,
performs a single random modification from a set of
predefined types,
and outputs the result.
The program uses regular expressions to group tokens int
six categories:
identifiers (including reserved words),
horizontal white space (spaces and tabs),
integer constants,
floating point constants,
group delimiters (brackets, square and curly braces), and
operators (including the multi-character operators of our chosen languages).

Based on this categorization,
we defined the following five modification types.
\begin{description}
\item [Identifier Substitution --- IdSub]
A single randomly chose identifier is replaced with another one,
randomly-chosen from the program's tokens.
This change can simulate absent-mindedness, a semantic error, or
a search-and-replace or refactoring operation gone awry.
\item [Integer Perturbation --- IntPert]
The value of a randomly chosen integer constant
is randomly perturbed by $1$ or $-1$.
This change simulates off-by-one errors.
\item [Random Character Substitution --- RandCharSub]
A single randomly chosen character (byte) within a randomly chose token
is substituted with a random byte.
This change simulates a typo or error in a complex editing command.
\item [Similar Token Substitution --- SimSub]
A single randomly chosen token
that is not a space character or a group delimiter
is substituted with another token of the same category,
randomly chosen from the program's source code.
This change simulates absent-mindedness and semantic errors.
\item [Random Token Substitution --- RandTokenSub]
A single randomly chosen non-space token
is substituted with another token of the same category,
This change can simulate most of the previously described errors.
\end{description}

Most fuzzing operations are implemented in a Monte Carlo style:
tokens are randomly chosen until they match the operation's constraints.
To aid the reproducibility of our results,
the fuzzer's random number generator is seeded with a constant value,
offset by another constant argument that is incremented on each successive run
and a hash value of the specified fuzzing operation.
Thus each time the fuzzer is executed with the same parameters it
produces the same results.

To run our tasks we created for each one of our languages two methods.
One compiles the source code into an executable program.
For interpreted languages this method checks the program's syntactic validity.
The aim of this ``compilation'' method is to test for errors that can
be statically detected before deployment.
The second method invokes (if required) the particular language's
runtime environment to run the executable program
(or the source code for interpreted languages),
and stores the results into a file.

A separate driver program compiled and run all the tasks from the ten
languages introducing fuzz into their source code.
As a task's programs written in different languages produce slightly
different results,
the driver program first runs an unmodified version of each task
to determine its expected output.
Output that diverges from it is deemed to be incorrect.
The running of each fuzzed task can fail in one of four successive
phases.
\begin{description}
\item[Fuzzing]
The fuzzer may fail to locate source code tokens that match the
constraints of a particular fuzzing operation.
This was a rare phenomenon, which mainly occurred in very short programs.
\item[Compilation --- com]
The program fails to compile (or syntax check),
as indicated through the compiler or interpreter's exit code.
In one particular case a fuzz
(a substitution of a closing bracket with {\tt func\_t})
caused an Objective C task's compiler
to enter into an infinite loop,
producing a 5{\sc gb} file of error messages.
We side-stepped this problem when we decided to remove Objective C from
the languages we tested.
\item[Execution --- run]
The program terminates successfully (without crashing),
as indicated by the program's exit code.
We had cases where the fuzzed code failed to terminate.
We detected those cases by imposing a 5s timeout on the time a program
was allowed to execute.
\item[Output Validity --- out]
The fuzzed program is producing the results different from those of
the original one.
\end{description}

The driver program run a complete fuzz, compile, run, verify cycle
for each of the five fuzz operations 50 times for
each task and each supported language.
We collected the results of these operations in an 86 thousand row
table,
which we analyzed through simple scripts.

\section{Results} % {{{1
\label{sec:results}

\begin{table*}
\begin{center}
\begin{tabular}{ l r r r r r r r r r r r r r r r r r r }
 & \multicolumn{3}{c}{IdSub (\%)} & \multicolumn{3}{c}{IntPert (\%)} & \multicolumn{3}{c}{RandCharSub (\%)} & \multicolumn{3}{c}{RandTokenSub (\%)} & \multicolumn{3}{c}{SimSub (\%)}\\
           & com  & run  & out  & com   & run  & out   & com  & run  & out  & com  & run  & out  & com  & run  & out\\
\hline
C          & 16.6 & 14.0 & 10.4 & 100.0 & 52.6 & 18.6  & 7.3  & 7.3  & 6.1  & 5.4  & 5.1  & 3.4  & 20.6 & 16.6 & 9.6 \\
C++        & 5.6  & 4.9  & 3.9  & 91.7  & 47.2 & 6.2   & 3.7  & 3.7  & 3.4  & 2.6  & 2.4  & 1.3  & 8.3  & 7.1  & 3.1 \\
C\#        & 6.9  & 6.9  & 6.6  & 69.8  & 65.2 & 11.9  & 4.0  & 4.0  & 3.9  & 3.0  & 3.0  & 2.7  & 7.7  & 7.4  & 6.0 \\
Haskell    & 4.0  & 1.9  & 0.4  & 89.4  & 84.0 & 10.3  & 3.7  & 3.4  & 2.3  & 3.5  & 3.2  & 1.8  & 13.4 & 11.2 & 2.1 \\
Java       & 5.1  & 3.6  & 2.9  & 100.0 & 82.5 & 20.3  & 3.1  & 3.0  & 2.9  & 2.3  & 1.9  & 1.7  & 7.9  & 6.4  & 3.1 \\
Javascript & 65.9 & 18.4 & 5.7  & 100.0 & 79.7 & 6.0   & 30.9 & 9.6  & 7.9  & 15.0 & 5.7  & 1.9  & 57.2 & 22.9 & 5.1 \\
PHP        & 56.4 & 34.1 & 3.0  & 99.2  & 87.5 & 16.2  & 37.4 & 32.7 & 1.9  & 25.7 & 23.7 & 1.1  & 46.2 & 39.7 & 1.4 \\
Perl       & 57.9 & 29.3 & 7.4  & 100.0 & 91.0 & 22.1  & 15.1 & 11.6 & 6.7  & 18.2 & 14.2 & 4.9  & 44.3 & 27.3 & 10.5 \\
Python     & 43.3 & 17.3 & 12.6 & 100.0 & 75.3 & 14.4  & 18.3 & 6.9  & 6.1  & 20.7 & 10.6 & 5.7  & 45.2 & 23.6 & 10.0 \\
Ruby       & 52.3 & 11.8 & 9.1  & 100.0 & 91.1 & 31.6  & 27.6 & 14.7 & 12.3 & 33.4 & 15.8 & 11.1 & 58.0 & 27.0 & 16.1 \\
\hline
\textbf{Mean} & 30.6 & 14.0 & 6.2  & 95.1  & 74.5 & 15.8  & 15.1 & 9.7  & 5.3  & 12.8 & 8.5  & 3.5  & 30.3 & 18.7 & 6.6 \\
\end{tabular}
\end{center}
\caption{Aggregated results per language}
\label{tbl:aggregated-per-language}
\end{table*}

\section{Discussion} % {{{1
\label{sec:discussion}

\section{Related Work} % {{{1
\label{sec:related}

Fuzzing as a technique to investigate the reliability of software
was first proposed in an article by Miller and his colleagues~\cite{MFS90}.

In this paper they tested the common collection of {\sc unix}
utilities in various operating systems and architectures and discrovered that
25-33\% of these, are crashing under certain conditions. To perform these tests
they implemented automation shell scripts and a fuzzer, a program that generated
random character sequences according to certain specifications.

Fuzzing is used mainly to detect software security vulnerabilities and 
improve overall reliability \cite{TJC08,GODE07}. Several tools and techniques \cite{WWGZ11}
have been developed, introducing concept like \textit{directed fuzz testing} \cite{GLRI09}.

Our approach fuzzes the program with pseudo-random
perturbations based in same cases on knowledge of
lexical tokens. This can result in a large number of failures.

An alternative approach, {\em grammar-based white box fuzzing} \cite{God08},
takes into account the input language's grammar to fuzz the input in
ways that are syntactically correct.
This results in a higher rate of successful fuzzing and the location
of deeper problems.

Another interesting approach is H-fuzzing \cite{ZWZH11}, which is a heuristic method that examines the execution paths
of the program to achieve higher path coverage. All these approaches are based on the fact that it is practically impossible to detect all
execution paths and all program inputs to fully test the validity and the reliability of a program.

Random Testing \cite{HAM06} seems to be promising solution that can partially deal with this problem,
but still it is not widely adopted outside the academic fields \cite{GGBO07}, since the techniques it 
introduces are difficult to be applied in complex systems and achieve good code coverage, if so at a significant cost \cite{RAWO06}.

The above case is made worse with the use of complex refactorings \cite{Fow00} in day-to-day programming. Refactorings are beneficial, but they are programs integrated in {\sc ide}s and also have bugs \cite{DDGM07}. Every bug results in corrupted code, which is very difficult to detect, especially for the case of dynamic langugages \cite{SCHA12,FFM11}.

Our experiment aims to exhibit the fault tolerance \cite{LYU95,KOKR07} of each language and use their features such as 
their type systems to our advantage. 

Type systems are usually realized as type checkers in compilers and linkers of programming languages. They are categorized as static and dynamic. \textit{Static typing} implies static type checking. In other words, the ability of a programming language's compiler to know all the data types that are used in a program, and guarantee that they are correct at compile time. \textit{Dynamic typing} is exactly the opposite. The languages that are dynamically checked perform run-time checks in order to ensure consistent type use. 

In addition, a programming language is called safe, when it protects its own abstractions. For example, a programming language that provides an abstraction for arrays, is considered safe, if it performs boundary checking upon access. Consequently, Java is a statically checked and safe programming language, while C/C++ are statically checked and unsafe. On the other hand, Perl and Lisp are dynamically checked and safe \cite{Pie02}.

Statically typed languages are less tolerant to our approach, since the compiler catches all syntax or type errors and fail the compilation process. Dynamic languages like python, perl and ruby are more tolerant and usually they fail upon execution or produce corrupted output.

\section{Conclusions} % {{{1
\label{sec:conclusions}


\acks

We would like to thank Florents Tselai for significant
help in the porting and implementation of the
{\em Rosetta Code} tasks in our environment.


\paragraph{Code Availability} The source code for
the implemented tasks,
the fuzzer,
the language-specific methods, and
the driver are maintained on GitHub, and
will be made publicly available as open source software
before the paper is published.
% XXX STEREO

% We recommend abbrvnat bibliography style.

\bibliographystyle{abbrvnat}
\bibliography{fuzzer}

% The bibliography should be embedded for final submission.

%\begin{thebibliography}{}
%\softraggedright

%\bibitem[Smith et~al.(2009)Smith, Jones]{smith02}
%P. Q. Smith, and X. Y. Jones. ...reference text...

%\end{thebibliography}

\end{document}
