% vim: spell
\documentclass[10pt]{sigplanconf}

% The following \documentclass options may be useful:
%
% 10pt     To set in 10-point type instead of 9-point.
% 11pt     To set in 11-point type instead of 9-point.
% authoryear  To obtain author/year citation style instead of numeric.

\usepackage{amsmath}
\usepackage{bbding}
\usepackage{pifont}
\usepackage{url}

\begin{document}

\conferenceinfo{PLATEAU '12}{October 21, Tucson (AZ)} 
\copyrightyear{2012} 
\copyrightdata{[to be supplied]}

%\titlebanner{banner above paper title}    % These are ignored unless
%\preprintfooter{short description of paper}  % 'preprint' option specified.

\title{Programming Languages vs. Fat Fingers}
\subtitle{An Empirical Study}

\authorinfo{Diomidis Spinellis\and Vassilios Karakoidas}
      {Athens University of Economics and Business}
      {\{dds, bkarak\}@aueb.gr}

\maketitle

\begin{abstract}
We explore how programs written in ten popular programming languages
are affected by small changes of their source code.
This allows us to analyze the extend to which these languages
can detect simple errors at compile or at run time.
Our study is based on a large corpus of programs written in several programming
languages systematically perturbed using a mutation-based fuzz generator.
We found ... % XXX
\end{abstract}

\category{CR-number}{subcategory}{third-level}

\terms
term1, term2

\keywords
Programming Languages, Compiler, Fuzzing, Unit Testing

\section{Introduction} % {{{1

\cite{C99}

\begin{itemize}
	\item The common case in this experiment is the human factor, how the program responds to human errors (syntax, mistypes).
	\item The fuzzer that messes with variables and their values follows the standard pattern that is used to test applications and enhance their security against input validation attacks.
	\item The work presented here, could measure the fault tolerance for these languages. There is a generic term ``software fault tolerance''.
	\item Refactoring tools may act as fuzzers (they are programs and we have to assume that they have bugs), thus we are not measuring the resiliense of the programming languages against the human factor, we do it even against automated code refactoring tools.
\end{itemize}


\section{Methodology} % {{{1

We selected the languages to test based on a number of sources
collated in an {\sc ieee} Spectrum article \cite{Kin11}:
an index created by
{\sc tiobe}\footnote{\url{http://www.tiobe.com/index.php/content/paperinfo/tpci/index.html}} (a software research firm),
the number of book titles listed on Powell's Books,
references in online discussions on {\sc irc}, and
number of job posts on Craigslist.
From the superset of the popular languages listed in those
sources we excluded
Actionscript, Visual Basic, {\sc sql}, Objective C, and the Unix shell,
because the corresponding programs or infrastructure would not match our methods.
According to the source composition,
our language coverage ranges from 71\% to 86\% of all languages.
The list of languages we used in our study and the
particular implementations we used are listed in
Table~\ref{tab:langs}.

% FuzzIdentifierSubstitution', 'FuzzIntegerPerturbation', 'FuzzRandomCharacterSubstitution', 'FuzzRandomTokenSubstitution', 'FuzzSimilarSubstitution'
\begin{description}
	\item [IdentifierSubstitution (IdSub)]
	\item [IntegerPerturbation (IntPert)]
	\item [RandomCharacterSubstitution (RandCharSub)]
	\item [RandomTokenSubstitution (RandTokenSub)] Replace one random token with another. This simulates abbsent-mindendness
	\item [SimilarSubstitution (SimSub)]
\end{description}

the fuzzer result is deterministic? Each time we executed the fuzzer to the source, it produces the same permutation of the code?

We measure:

\begin{itemize}
	\item Compilation Failure: if the a project with the specified fuzzer fails to compile.
	\item Execution Failure: if the program is executing correctly (without crashing)
	\item Result Error: if the program is not producing the proper result
\end{itemize}

\begin{table}
\begin{center}
\caption{Tested languages.}
\label{tab:langs}
\begin{tabular}{ l l}
 \hline
Language & Implementation \\
\hline
C 			& gcc 4.4.5 \\
C++ 		& g++ 4.4.5 \\
C\# 		& mono 2.6.7, CLI v2.0 \\
Haskell 	& ghc 6.12.1 \\
Java 		& OpenJDK 1.6.0\_18 \\
Javascript 	& spidermonkey 1.8.0 \\
PHP 		& PHP 5.3.3-7 \\
Perl 		& perl 5.10.1 \\
Python 		& python 2.6.6 \\
Ruby 		& ruby 1.5.8 \\
\hline
\end{tabular}
\end{center}
\end{table}

\begin{table}
\begin{center}
\caption{List of selected Rosseta Code tasks.}
\label{tab:Tasks}
\begin{tabular}{ l p{5cm}}
 \hline
Task Name & Description\\
\hline
AccumFactory & Accumulator Generator as described by Paul Graham\footnote{\url{http://paulgraham.com/accgensub.html}}\\
Beers \\
Dow & Detects all years (in a range) that christmas is a Sunday.\\
FlatList & Flattens a series of nested lists.\\
FuncComp & Implementation of mathematical function composition.\\
Horner & Horner's rule for polynomial evaluation.\\
Hello & A typical ``hello, world!'' program.\\
Mult & Ethiopian Multiplication. A method to multiply integers using only addition, doubling and halving.\\
MutRecursion & Hofstadter Female and Male sequences\footnote{\url{http://en.wikipedia.org/wiki/Hofstadter_sequence#Hofstadter_Female_and_Male_sequences}}.\\
ManBoy & \textit{Man or Boy} test as proposed by Don Knuth.\\
Power & Implementation of Power sets\footnote{\url{http://en.wikipedia.org/wiki/Power_set}}.\\
Substring & A program that count the occurrences of a substring.\\
Tokenizer & A string tokenizing program.\\
ZigZag & Implementation of a program that produces a zig-zag array\footnote{\url{http://rosettacode.org/wiki/Zig-zag_matrix}}.\\
\hline
\end{tabular}
\end{center}
\end{table}

\section{Results} % {{{1

\begin{table*}
% ['c', 'cpp', 'cs', 'hs', 'java', 'js', 'php', 'pl', 'py', 'rb']
\begin{center}
\caption{Language Compatibility per Task.}
\label{tbl:lang-compatibility}
\begin{tabular}{l c c c c c c c c c c}
 \hline
 & C & C++ & C\# & Haskell & Java & Javascript & PHP & Perl & Python & Ruby\\
\hline
AccumFactory & \ding{51} & \ding{51} & \ding{51} & \ding{51} & \ding{51} & \ding{51} & \ding{51} & \ding{51} & \ding{51} & \ding{51} \\
Hello & \ding{51} & \ding{51} & \ding{51} & \ding{51} & \ding{51} & \ding{51} & \ding{51} & \ding{51} & \ding{51} & \ding{51} \\
FlatList & \ding{51} & \ding{55} & \ding{55} & \ding{51} & \ding{55} & \ding{51} & \ding{51} & \ding{51} & \ding{51} & \ding{51} \\
Power & \ding{51} & \ding{51} & \ding{55} & \ding{51} & \ding{51} & \ding{51} & \ding{51} & \ding{51} & \ding{51} & \ding{51} \\
ZigZag & \ding{51} & \ding{51} & \ding{55} & \ding{51} & \ding{51} & \ding{55} & \ding{51} & \ding{51} & \ding{51} & \ding{51} \\
FuncComp & \ding{51} & \ding{51} & \ding{51} & \ding{51} & \ding{51} & \ding{51} & \ding{51} & \ding{51} & \ding{51} & \ding{51} \\
Substring & \ding{51} & \ding{51} & \ding{51} & \ding{55} & \ding{51} & \ding{51} & \ding{51} & \ding{51} & \ding{51} & \ding{51} \\
ManBoy & \ding{51} & \ding{51} & \ding{51} & \ding{51} & \ding{51} & \ding{51} & \ding{51} & \ding{51} & \ding{51} & \ding{51} \\
Beers & \ding{51} & \ding{51} & \ding{51} & \ding{51} & \ding{51} & \ding{51} & \ding{51} & \ding{51} & \ding{51} & \ding{51} \\
Tokenizer & \ding{51} & \ding{51} & \ding{51} & \ding{55} & \ding{51} & \ding{51} & \ding{51} & \ding{51} & \ding{51} & \ding{51} \\
Horner & \ding{51} & \ding{51} & \ding{51} & \ding{51} & \ding{51} & \ding{51} & \ding{51} & \ding{51} & \ding{51} & \ding{51} \\
MutRecursion & \ding{51} & \ding{51} & \ding{51} & \ding{51} & \ding{51} & \ding{51} & \ding{51} & \ding{51} & \ding{51} & \ding{51} \\
Dow & \ding{51} & \ding{51} & \ding{51} & \ding{51} & \ding{51} & \ding{51} & \ding{51} & \ding{51} & \ding{51} & \ding{51} \\
Mult & \ding{51} & \ding{51} & \ding{51} & \ding{51} & \ding{51} & \ding{51} & \ding{51} & \ding{51} & \ding{51} & \ding{51} \\
 & 14 & 13 & 11 & 12 & 13 & 13 & 14 & 14 & 14 & 14 \\
\hline
\end{tabular}
\end{center}
\end{table*}

\begin{table*}
\begin{center}
\caption{Aggregated results per language}
\label{tbl:aggregated-per-language}
\begin{tabular}{ l r r r r r r r r r r r r r r r r r r }
 \hline
 & \multicolumn{3}{c}{IdSub (\%)} & \multicolumn{3}{c}{IntPert (\%)} & \multicolumn{3}{c}{RandCharSub (\%)} & \multicolumn{3}{c}{RandTokenSub (\%)} & \multicolumn{3}{c}{SimSub (\%)}\\
 & com & run & out & com & run & out & com & run & out & com & run & out & com & run & out\\
\hline
C & 16 & 84 & 74 & 100 & 52 & 35 & 7 & 100 & 84 & 5 & 94 & 66 & 20 & 80 & 57 \\
C++ & 5 & 87 & 79 & 91 & 51 & 13 & 3 & 100 & 92 & 2 & 94 & 52 & 8 & 86 & 44 \\
C\# & 6 & 100 & 95 & 69 & 93 & 18 & 4 & 100 & 96 & 3 & 100 & 90 & 7 & 96 & 80 \\
Haskell & 4 & 48 & 23 & 89 & 93 & 12 & 3 & 92 & 66 & 3 & 91 & 54 & 13 & 83 & 18 \\
Java & 5 & 69 & 80 & 100 & 82 & 24 & 3 & 95 & 95 & 2 & 81 & 92 & 7 & 81 & 48 \\
Javascript & 65 & 27 & 31 & 100 & 79 & 7 & 30 & 31 & 82 & 15 & 38 & 32 & 57 & 40 & 22 \\
PHP & 56 & 60 & 8 & 99 & 88 & 18 & 37 & 87 & 5 & 25 & 92 & 4 & 46 & 86 & 3 \\
Perl & 57 & 50 & 25 & 100 & 91 & 24 & 15 & 76 & 58 & 18 & 77 & 34 & 44 & 61 & 38 \\
Python & 43 & 39 & 72 & 100 & 75 & 19 & 18 & 37 & 89 & 20 & 51 & 54 & 45 & 52 & 42 \\
Ruby & 52 & 22 & 77 & 100 & 91 & 34 & 27 & 53 & 83 & 33 & 47 & 69 & 57 & 46 & 59 \\
\hline
\end{tabular}
\end{center}
\end{table*}

\section{Discussion} % {{{1

\section{Related Work} % {{{1

Fuzzing as a technique to investigate the reliability of software
was first proposed in an article by Miller and his colleagues~\cite{MFS90}.

In that they describe how ... % XXX 

Fuzzing ... \cite{TJC08}, \cite{WWGZ11}

Out approach fuzzes the program with pseudo-random
perturbations based in same cases on knowledge of
lexical tokens.
This can result in a large number of failures.
An alternative approach, {\em grammar-based white box fuzzing}~\cite{God08},
takes into account the input language's grammar to fuzz the input in
ways that are syntactically correct.
This results in a higher rate of successful fuzzing and the location
of deeper problems.

Testing ... \cite{HAM06}

Fault Tolerance ... \cite{KOKR07}, \cite{LYU95}

\section{Conclusions} % {{{1

\acks

STEREO
Acknowledgments, if needed.

% We recommend abbrvnat bibliography style.

\bibliographystyle{abbrvnat}
\bibliography{fuzzer}

% The bibliography should be embedded for final submission.

%\begin{thebibliography}{}
%\softraggedright

%\bibitem[Smith et~al.(2009)Smith, Jones]{smith02}
%P. Q. Smith, and X. Y. Jones. ...reference text...

%\end{thebibliography}

\end{document}
