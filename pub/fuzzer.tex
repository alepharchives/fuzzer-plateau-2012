% vim: spell
\documentclass[10pt]{sigplanconf}

% The following \documentclass options may be useful:
%
% 10pt          To set in 10-point type instead of 9-point.
% 11pt          To set in 11-point type instead of 9-point.
% authoryear    To obtain author/year citation style instead of numeric.

\usepackage{amsmath}
\usepackage{url}

\begin{document}

\conferenceinfo{PLATEAU '12}{October 21, Tucson (AZ)} 
\copyrightyear{2012} 
\copyrightdata{[to be supplied]}

%\titlebanner{banner above paper title}        % These are ignored unless
%\preprintfooter{short description of paper}   % 'preprint' option specified.

\title{Still Thinking About it!}
%\subtitle{Subtitle Text, if any}

\authorinfo{Diomidis Spinellis\and Vassilios Karakoidas}
           {Athens University of Economics and Business}
           {\{dds, bkarak\}@aueb.gr}

\maketitle

\begin{abstract}
This is the text of the abstract.
\end{abstract}

\category{CR-number}{subcategory}{third-level}

\terms
term1, term2

\keywords
Programming Languages, Compiler, Fuzzing, Unit Testing

\section{Introduction} % {{{1

\cite{C99}

\begin{itemize}
	\item The common case in this experiment is the human factor, how the program responds to human errors (syntax, mistypes).
	\item The fuzzer that messes with variables and their values follows the standard pattern that is used to test applications and enhance their security against input validation attacks.
	\item The work presented here, could measure the fault tolerance for these languages. There is a generic term ``software fault tolerance'', which describes te
	\item Refactoring tools may act as fuzzers (they are programs and we have to assume that they have bugs), thus we are not measuring the resiliense of the programming languages against the human factor, we do it even against automated code refactoring tools.
\end{itemize}


\section{Methodology} % {{{1
\begin{table}
\begin{center}
\caption{Tested languages.}
\label{tab:langs}
\begin{tabular}{ l l}
 \hline
Language & Implementation \\
\hline
C 		& gcc 4.4.5 \\
C++ 		& g++ 4.4.5 \\
C\# 		& mono 2.6.7, CLI v2.0 \\
Haskell 	& ghc 6.12.1 \\
Java 		& OpenJDK 1.6.0\_18 \\
Javascript 	& spidermonkey smjs 1.8.0 \\
PHP 		& PHP 5.3.3-7 \\
Perl 		& perl 5.10.1 \\
Python 		& python 2.6.6 \\
Ruby 		& ruby 1.5.8 \\
\hline
\end{tabular}
\end{center}
\end{table}

We selected the languages to test based on a number of sources
collated in an {\sc ieee} Spectrum article \cite{Kin11}:
an index created by
{\sc tiobe}\footnote{\url{http://www.tiobe.com/index.php/content/paperinfo/tpci/index.html}} (a software research firm),
the number of book titles listed on Powell's Books,
references in online discussions on {\sc irc}, and
number of job posts on Craigslist.
From the superset of the popular languages listed in those
sources we excluded
Actionscript, Visual Basic, {\sc sql}, Objective C, and the Unix shell,
because the corresponding programs or infrastructure would not match our methods.
According to the source composition,
our language coverage ranges from 71\% to 86\% of all languages.
The list of languages we used in our study and the
particular implementations we used are listed in
Table~\ref{tab:lang}.

\begin{table}
\begin{center}
\caption{List of selected Rosseta Code tasks.}
\label{tab:Tasks}
\begin{tabular}{ l p{4cm}}
 \hline
Task Name & Description\\
\hline
AccumFactory & \\
Dow & \\
FlatList & \\
FuncComp & \\
Horner & \\
Hello & A typical ``hello, world!'' program\\
Mult & \\
MutRecursion & \\
ManBoy & \\
\hline
\end{tabular}
\end{center}
\end{table}


Fuzzing types
\begin{itemize}
\item Replace one random token with another.
This simulates abbsent-mindendness 
\end{itemize}

the fuzzer result is deterministic? Each time we executed the fuzzer to the source, it produces the same permutation of the code?

We measure:

\begin{itemize}
	\item Compilation Failure: if the a project with the specified fuzzer fails to compile.
	\item Execution Failure: if the program is executing correctly (without crashing)
	\item Result Error: if the program is not producing the proper result
\end{itemize}

\section{Results} % {{{1

\section{Discussion} % {{{1

\section{Related Work} % {{{1

Fuzzing ... \cite{TJC08}, \cite{WWGZ11}

Out approach fuzzes the program with pseudo-random
perturbations based in same cases on knowledge of
lexical tokens.
This can result in a large number of failures.
An alternative approach, {\em grammar-based white box fuzzing}~\cite{Cog08},
takes into account the input language's grammar to fuzz the input in
ways that are syntactically correct.
This results in a higher rate of successful fuzzing and the location
of deeper problems.


Testing ... \cite{HAM06}

Fault Tolerance ... \cite{KOKR07}, \cite{LYU95}

\section{Conclusions} % {{{1

\acks

STEREO
Acknowledgments, if needed.

% We recommend abbrvnat bibliography style.

\bibliographystyle{abbrvnat}
\bibliography{fuzzer}

% The bibliography should be embedded for final submission.

%\begin{thebibliography}{}
%\softraggedright

%\bibitem[Smith et~al.(2009)Smith, Jones]{smith02}
%P. Q. Smith, and X. Y. Jones. ...reference text...

%\end{thebibliography}

\end{document}
