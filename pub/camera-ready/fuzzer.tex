% vim: spell
\documentclass[10pt]{sigplanconf}

% The following \documentclass options may be useful:
%
% 10pt  To set in 10-point type instead of 9-point.
% 11pt  To set in 11-point type instead of 9-point.
% authoryear To obtain author/year citation style instead of numeric.

\usepackage{amsmath}
\usepackage{amssymb}
\usepackage{bbding}
\usepackage{pifont}
\usepackage{url}
\usepackage{graphicx}
\usepackage{subfigure}
\usepackage{lscape}

\begin{document}

\conferenceinfo{PLATEAU'12,} {October 21, 2012, Tucson, Arizona, USA.}
\CopyrightYear{2012}
\copyrightdata{978-1-4503-1631-6/12/10}
\toappear{Permission to make digital or hard copies of all or part of this work for personal or classroom use is granted without fee provided that copies are not made or distributed for profit or commercial advantage and that copies bear this notice and the full citation on the first page. To copy otherwise, or republish, to post on servers or to redistribute to lists, requires prior specific permission and/or a fee. PLATEAU’12, October 21, 2012, Tucson, Arizona, USA.
Copyright © 2012 ACM 978-1-4503-1631-6/12/10...\$15.00.}

\title{Comparative Language Fuzz Testing}
\subtitle{Programming Languages vs. Fat Fingers}

\authorinfo{Diomidis Spinellis\and Vassilios Karakoidas\and Panos Louridas}
  {Athens University of Economics and Business}
  {\{dds, bkarak, louridas\}@aueb.gr}

\maketitle

\begin{abstract}
We explore how programs written in ten popular programming languages
are affected by small changes of their source code.
This allows us to analyze the extend to which these languages
allow the detection of simple errors at compile or at run time.
Our study is based on a diverse corpus of programs written in several programming
languages systematically perturbed using a mutation-based fuzz generator.
The results we obtained
prove that languages with weak type systems are significantly
likelier than languages that enforce strong typing to let fuzzed programs
compile and run, and, in the end, produce erroneous results.
More importantly, our study also demonstrates the potential of comparative
language fuzz testing for evaluating programming language designs.
\end{abstract}

 \category{D.3.0}{Programming Languages}{General}

\terms
Reliability, Experimentation, Measurement

\keywords
Programming Languages; Fuzzing; Comparison; Rosetta Stone

\section{Introduction} % {{{1
\label{sec:intro}
A substitution of a comma with a period in project Mercury's working
{\sc fortran} code compromised the accuracy of the results,
rendering them unsuitable for longer orbital missions \cite{Brad89,Neu95}.
How probable are such events and how does a programming language's
design affect their likelihood and severity?

To study these questions we chose ten popular programming languages,
and a corpus of programs written in all of them.
We then constructed a source code mutation {\em fuzzer}:
a tool that systematically introduces diverse random perturbations
into the program's source code.
Finally, we applied the fuzzing tool on the source code corpus
and examined whether the resultant code had errors that
were detected at compile or run time, and whether it produced
erroneous results.

In practice,
the errors that we artificially introduced into the source code can
crop up in a number of ways.
Mistyping---the ``fat fingers'' syndrome---is one plausible source.
Other scenarios include
absent-mindedness,
automated refactorings \cite{Fow00} gone awry
(especially in languages where such tasks cannot be reliably implemented),
unintended consequences from complex editor commands or
search-and-replace operations,
and even the odd cat walking over the keyboard.

The contribution of our work is twofold.
First, we describe a method for systematically evaluating the tolerance
of source code written in diverse programming languages to a particular
class of errors.
In addition, we apply this method to numerous tasks written in ten popular
programming languages,
and by analyzing tens of thousands of cases we present an overview of
the likelihood and impact of these errors among ten popular languages.

In the remainder of this paper we
outline our methods (Section~\ref{sec:method}),
present and discuss our findings (Section~\ref{sec:results}),
compare our approach against related work (Section~\ref{sec:related}),
and conclude with proposals for further study (Section~\ref{sec:conclusions}).

\section{Methodology} % {{{1
\label{sec:method}

\begin{table}
\begin{center}
\begin{tabular}{ l l}
Language & Implementation \\
\hline
C 			& gcc 4.4.5 \\
C++ 		& g++ 4.4.5 \\
C\# 		& mono 2.6.7, CLI v2.0 \\
Haskell 	& ghc 6.12.1 \\
Java 		& OpenJDK 1.6.0\_18 \\
JavaScript 	& spidermonkey 1.8.0 \\
{\sc php} 		& {\sc php} 5.3.3-7 \\
Perl 		& perl 5.10.1 \\
Python 		& python 2.6.6 \\
Ruby 		& ruby 1.8.7 \\
\end{tabular}
\end{center}
\caption{Studied languages.}
\label{tab:langs}
\end{table}

We selected the languages to test based on a number of sources
collated in an {\sc ieee} Spectrum article \cite{Kin11}:
an index created by
{\sc tiobe}\footnote{\url{http://www.tiobe.com/index.php/content/paperinfo/tpci/index.html}} (a software research firm),
the number of book titles listed on Powell's Books,
references in online discussions on {\sc irc}, and
the number of job posts on Craigslist.
From the superset of the popular languages listed in those
sources we excluded some languages for the following reasons.
\begin{description}
\item[Actionscript, Visual Basic] Both languages
required a proprietary compiler and runtime environment,
which were not available on our system.
\item[{\sc sql}, Unix shell] Lack of implementations of the programs
we could test.
\item[Objective C] Problems with the requisite runtime environment:
missing libraries, incompatible runtime frameworks,
and lack of familiarity with the system.
\end{description}

The list of the ten languages we adopted for our study and the
particular implementations we used are listed in
Table~\ref{tab:langs}.
According to the source of the popularity index,
the coverage of the languages we selected over all languages
ranges from 71\% to 86\%.

We obtained fragments of source code executing the same task in all of
our study's ten languages from
{\em Rosetta Code},\footnote{\url{http://rosettacode.org/}}
a so-called programming chrestomathy site,
implemented as a wiki.
In the words of its creators,
the site aims to present code for the same task in as many languages as possible,
thus demonstrating their similarities and differences and
aiding persons with a grounding in one approach to a problem in learning another.
At the time of our writing {\em Rosetta Code}
listed 600 tasks and code in 470 languages.
However, most of the tasks are presented only in a subset of those languages.

\begin{table}
\begin{center}
\begin{tabular}{ l p{5cm}}
Task Name & Description\\
\hline
AccumFactory & A function that takes a number $n$ and returns a function that acts as an accumulator and also accepts a number. Each function should return the sum of the numbers added to the accumulator so far.\\
Beers & Print the ``99 bottles of beer on the wall'' song.\\
Dow & Detects all years in a range in which Christmas falls on a Sunday.\\
FlatList & Flattens a series of nested lists.\\
FuncComp & Implementation of mathematical function composition.\\
Horner & Horner's Method for polynomial evaluation.\\
Hello & A typical ``hello, world!'' program.\\
Mult & Ethiopian Multiplication: a method to multiply integers using only addition, doubling and halving.\\
MutRecursion & Hofstadter's Female and Male sequence~\cite{Hof89}.\\
ManBoy & A test to distinguish compilers that correctly implement
recursion and non-local references from those that do not~\cite{Knu64}.  \\
Power & Calculation of a set's $S$ power set: the set of all subsets of $S$.\\
Substring & Count the occurrences of a substring.\\
Tokenizer & A string tokenizing program.\\
ZigZag & Produce a square arrangement of the first $N^2$ integers,
where the numbers increase sequentially in a zig-zag along the anti-diagonals of the array.\\
\end{tabular}
\end{center}
\caption{List of the selected {\em Rosseta Code} tasks.}
\label{tab:tasks}
\end{table}

We selected our tasks from {\em Rosetta Code} through the following process.
First, we downloaded the listing of all available tasks and
filtered it to create a list of task {\sc url}s.
We then downloaded the page for each task in MediWiki markup format,
located the headers for the languages in which that task was implemented, and
created a table containing tasks names and language names.
We joined that table with our chosen languages,
thus obtaining a count of the tasks implemented in
most of the languages in our set.
From that set we selected tasks that implemented diverse
non-trivial functionality,
and also, as a test case, the ``Hello, world!'' task.
The tasks we studied are listed in Table~\ref{tab:tasks}.

\begin{table*}
% ['c', 'cpp', 'cs', 'hs', 'java', 'js', 'php', 'pl', 'py', 'rb']
\begin{center}
\begin{tabular}{l r r r r r r r r r r   r}
 & C & C++ & C\# & Haskell & Java & JavaScript & {\sc php} & Perl & Python & Ruby & \textbf{Implemented}\\
 &   &     &     &         &      &            &     &      &        &      &  \textbf{Languages}\\
\hline
AccumFactory & 17 & 57 & 8 & 16 & 16 & 8 & 7 & 7 & 10 & 30 & 10 \\
Hello & 7 & 8 & 7 & 1 & 6 & 1 & 1 & 1 & 7 & 1 & 10 \\
FlatList & 118 & \ding{55} & 80 & 15 & 35 & 4 & 15 & 5 & 14 & 1 & 9 \\
Power & 27 & 77 & \ding{55} & 10 & 31 & 13 & 59 & 3 & 29 & 47 & 9 \\
ZigZag & 22 & 80 & 51 & 19 & 46 & \ding{55} & 31 & 15 & 13 & 14 & 9 \\
FuncComp & 60 & 34 & 18 & 4 & 32 & 6 & 7 & 9 & 3 & 7 & 10 \\
Substring & 21 & 21 & 35 & \ding{55} & 10 & 1 & 3 & 9 & 1 & 1 & 9 \\
ManBoy & 46 & 32 & 22 & 11 & 28 & 8 & 13 & 8 & 11 & 5 & 10 \\
Beers & 14 & 12 & 28 & 6 & 21 & 9 & 14 & 20 & 13 & 12 & 10 \\
Tokenizer & 22 & 15 & 16 & \ding{55} & 11 & 1 & 3 & 1 & 2 & 1 & 9 \\
Horner & 21 & 20 & 15 & 3 & 22 & 3 & 8 & 10 & 6 & 3 & 10 \\
MutRecursion & 29 & 35 & 31 & 8 & 20 & 18 & 22 & 28 & 4 & 8 & 10 \\
Dow & 23 & 17 & 17 & 7 & 13 & 5 & 9 & 17 & 7 & 4 & 10 \\
Mult & 31 & 53 & 61 & 14 & 40 & 25 & 32 & 23 & 41 & 25 & 10 \\
\hline
\textbf{Total lines} & 458 & 461 & 389 & 114 & 331 & 102 & 224 & 156 & 161 & 159 & \\
\end{tabular}
\end{center}
\caption{Lines of Code per Task and per Language, Unimplemented Tasks, and Implemented Languages per Task.}
\label{tbl:lang-compatibility}
\end{table*}

Unfortunately, many of the tasks listed on {\em Rosetta Stone} were
not in a form that would allow us to execute them as part of our study.
Many would not compile, others lacked a test harness to produce output,
and some required specific installed libraries or particular
new language features.
We tried to fix as many problems as possible, but in the end the tasks
we ended up using were not as large or diverse as we would have liked.
In addition, we were unable to implement some of the tasks in all our
chosen languages.
Tasks written in Objective-C, which was initially part of our language set,
proved particularly tricky to compile,
mainly because we found it difficult to automate their compilation and
running.
Key size metrics of the tasks and languages we tested are listed in
Table~\ref{tbl:lang-compatibility}.

We implemented a language-agnostic fuzzer as a Perl script
that reads a program,
splits it into tokens,
performs a single random modification from a set of
predefined types,
and outputs the result.
The program uses regular expressions to group tokens into
six categories:
identifiers (including reserved words),
horizontal white space (spaces and tabs),
integer constants,
floating point constants,
group delimiters (brackets, square and curly braces), and
operators (including the multi-character operators of our chosen languages).

Based on this categorization,
our intuition about common errors, and
what types of fuzzing could be implemented easily and portably across
diverse languages,
we defined five types of fuzzing modifications.
Given that we are not adding or removing elements,
all modifications correspond to an error of type
{\em presence: incorrect} according to the taxonomy proposed
by Ostrand and Weyuker~\cite{OW84}.
Although our choice could be improved by basing it on empirical data,
it turns out that our selection matches actual programmer errors.
For each item in the list below we indicate how other
researchers~\cite{End75,KNUTH89} categorize a programmer error
corresponding to such a modification.

\begin{description}
\item [Identifier Substitution --- IdSub]
A single randomly chosen identifier is replaced with another one,
randomly-chosen from the program's tokens.
This change can simulate absent-mindedness, a semantic error, or
a search-and-replace or refactoring operation gone awry.
\cite[B3.b]{End75},
\cite[B]{KNUTH89}
\item [Integer Perturbation --- IntPert]
The value of a randomly chosen integer constant
is randomly perturbed by $1$ or $-1$.
This change simulates off-by-one errors.
\cite[B4.b]{End75},
\cite[A]{KNUTH89}
\item [Random Character Substitution --- RandCharSub]
A single randomly chosen character (byte) within a randomly chosen token
is substituted with a random byte.
This change simulates a typo or error in a complex editing command.
\cite[C1]{End75},
\cite[T]{KNUTH89}
\item [Similar Token Substitution --- SimSub]
A single randomly chosen token
that is not a space character or a group delimiter
is substituted with another token of the same category,
randomly chosen from the program's source code.
This change simulates absent-mindedness and semantic errors.
\cite[B]{KNUTH89}
\item [Random Token Substitution --- RandTokenSub]
A single randomly chosen non-space token
is substituted with another token.
This change can simulate most of the previously described errors.
\cite[T, B]{KNUTH89}
\end{description}

Most fuzzing operations are implemented in a Monte Carlo fashion:
tokens are randomly chosen until they match the operation's constraints.
To aid the reproducibility of our results,
the fuzzer's random number generator is seeded with a constant value,
offset by another constant argument that is incremented on each successive run
and a hash value of the specified fuzzing operation.
Thus, each time the fuzzer is executed with the same parameters it
produces the same results.

To run our tasks we created for each one of our languages two methods.
One compiles the source code into an executable program.
For interpreted languages this method checks the program's syntactic validity.
The aim of this ``compilation'' method is to test for errors that can
be statically detected before deployment.
The second method invokes (if required) the particular language's
run time environment to run the executable program
(or the script for interpreted languages),
and stores the results into a file.

A separate driver program compiles and runs all the tasks from the ten
languages introducing fuzz into their source code.
As a task's programs written in different languages produce slightly
different results,
the driver program first runs an unmodified version of each task
to determine its expected output.
Output that diverges from it is deemed to be incorrect.
The running of each fuzzed task can fail in one of four successive
phases.
\begin{description}
\item[Fuzzing]
The fuzzer may fail to locate source code tokens that match the
constraints of a particular fuzzing operation.
This was a rare phenomenon, which mainly occurred in very short programs.
\item[Compilation --- com]
The program fails to compile (or syntax check),
as indicated through the compiler's or interpreter's exit code.
In one particular case a fuzz
(a substitution of a closing bracket with {\tt func\_t})
caused an Objective C task's compiler
to enter into an infinite loop,
producing a 5{\sc gb} file of error messages.
We side-stepped this problem when we decided to remove Objective C from
the languages we tested.
In another case the Haskell compiler entered an infinite loop.
To avoid such problems we imposed a 20s timeout on the compilation process.
\item[Execution --- run]
The program fails to terminate successfully,
as indicated by the program's exit code.
These failures included crashes.
We also had cases where the fuzzed code failed to terminate.
We detected those cases by imposing a 5s timeout on the time a program
was allowed to execute.
\item[Output Validity --- out]
The fuzzed program is producing results different from those of
the original one.
In contrast to a modern real-world scenario,
the programs we used lacked a test suite,
which we could employ to test a program independently from its productive
operation.
\end{description}

The driver program run a complete fuzz, compile, run, verify cycle
for each of the five fuzz operations 400 times for
each task and each supported language.
We collected the results of these operations in an 692,646 row table,
which we analyzed through simple scripts.
(The table's size is not round, because each task involves
fuzzing, compilation, running, and result comparison.
If a phase fails, the subsequent phases are not performed.)

\section{Results and Discussion} % {{{1
\label{sec:results}

In total we tested
% grep 'original COMPILE OK' run2.out | wc -l
136 task implementations
% grep FUZZ run2.out |wc -l
attempting 280,000 fuzzing operations,
% grep 'FUZZ OK' run2.out |wc -l
of which 261,667 (93\%) were successful.
% egrep -v 'original|prime' run2.out | fgrep 'COMPILE OK' | wc -l
From the fuzzed programs 90,166 (32\%)
compiled or were syntax-checked without a problem.
% egrep -v 'original|prime' run2.out | fgrep 'RUN OK' | wc -l
From those programs 60,126 (67\%, or 23\% of the fuzzed total) terminated successfully.
Of those 18,256 produced output identical to the reference one,
indicating that the fuzz was
inconsequential to the program's operation.
% egrep -v 'original|prime' run2.out | fgrep 'OUTPUT FAIL' | wc -l
The rest, 41,870 programs (70\% of those that run, 16\% of the fuzzed total),
compiled and run without a problem, but produced wrong output.

These aggregate results indicate that we chose an effective set
of fuzzing methods.
Syntax and semantic checking appear to be an effective but not
fail-safe method for detecting the fuzz errors we introduced,
as they blocked about two thirds of the fuzzed programs.
A large percentage of the programs also terminated successfully,
giving us in the end wrong results for 16\% of the programs.

This is worrying:
it indicates that a significant number of trivial changes in a program's source
code that can happen accidentally will not be caught at compile and run
time and will result in an erroneously operating program.
In an ideal case we might want program code to have enough redundancy
so that such small changes would result in an incorrect program
that would not compile.
However, as any user of {\sc raid} storage can attest,
redundancy comes at a cost.
Programming productivity in such a language would suffer as programmers
would have to write more code and keep in sync mutually dependent parts
of it.

\begin{figure*}
        \includegraphics[scale=0.60]{chart2}
        \caption{Failure modes for each phase per language and overall.}
        \label{fig:results}
\end{figure*}

The aggregate results per language are summarized in Figure~\ref{fig:results}
in the form of {\em failure modes}:
successful compilations or executions, which consequently failed to catch an
erroneous program and resulted in wrong results.
The rationale behind this depiction is that the later in the software
development life cycle an error is caught the more damaging it is.
The denominator used for calculating the percentages also includes
fuzzing operations that resulted in correct results.
Therefore, the numbers also reflect the programming language's
information density:
in our case the chance that a fuzz will not affect the program's operation.

The figure confirms a number of intuitive notions.
Languages with strong static typing \cite{Pie02} (Java, Haskell, C++)
caught more errors at compile time than languages
with weak or dynamic type systems
(Ruby, Python, Perl, {\sc php}, and JavaScript).
Somewhat predictably, C fell somewhere in the middle,
confirming a widely-held belief that its type system is not as strong
as many of its adherents (including this article's first author)
think it is.
However, C produced a higher number of run-time errors,
which in the end resulted in a rate of incorrect output
similar to that of the other strongly-typed languages.

A picture similar to that of compile-time errors
is also apparent for run time behavior.
Again, code written in weakly-typed languages is more probable to run without
a problem (a crash or an exception)
than code written in languages with a strong type system.
As one would expect these two differences result in a higher rate of
wrong output from programs written in languages with weak typing.
With an error rate of 36\% for {\sc php} against one of 8\% for
C++ and 10\% for C\#,
those writing safety-critical applications should carefully
weight the usability advantages offered by a weakly-type language,
like {\sc php}, against the increased risk that a typo
will slip undetected into production code.

As is often the case,
there is a trade-off between usability and reliability.
The adoption of languages in which typos can slip undetected
into production code is especially risky when a system also lacks
a thorough testing process.
Worryingly, in such a scenario, the effects of a typo may be
detected only in a program's real-life production run.

Overall, the figures for dynamic scripting languages show a far larger
degree of variation compared to the figures of the strongly static typed
ones.
This is probably a result of a higher level of experimentation
associated with scripting language features.

\begin{table*}
\begin{center}
\begin{tabular}{ l r r r|r r r|r r r|r r r|r r r}
& \multicolumn{3}{c}{IntPert (\%)} & \multicolumn{3}{c}{IdSub (\%)} & \multicolumn{3}{c}{SimSub (\%)} & \multicolumn{3}{c}{RandCharSub (\%)} & \multicolumn{3}{c}{RandTokenSub (\%)}\\ 
	       & com  & run  & out  & com  & run  & out  & com  & run  & out  & com & run   & out  & com  & run  & out\\
\hline													
C & 100.0 & 56.5 & 36.9  & 15.9 & 13.2 & 3.4  & 20.5 & 16.4 & 8.6  & 7.3 & 7.1 & 0.8  & 6.2 & 5.3 & 1.7 \\
C++ & 92.4 & 48.3 & 41.2  & 5.2 & 4.4 & 1.6  & 8.5 & 6.9 & 4.4  & 4.0 & 4.0 & 0.3  & 2.1 & 1.8 & 0.5 \\
C\# & 89.9 & 74.1 & 60.8  & 9.1 & 8.8 & 0.5  & 10.6 & 9.6 & 2.4  & 5.5 & 5.5 & 0.2  & 3.4 & 3.3 & 0.2 \\
Haskell & 89.2 & 83.5 & 72.2  & 5.2 & 3.3 & 2.4  & 13.5 & 11.4 & 9.4  & 2.9 & 2.6 & 0.6  & 3.8 & 3.3 & 2.0 \\
Java & 100.0 & 84.3 & 63.9  & 5.7 & 3.9 & 0.6  & 7.8 & 5.8 & 3.4  & 2.5 & 2.2 & 0.3  & 1.8 & 1.6 & 0.3 \\
Javascript & 100.0 & 80.3 & 74.9  & 66.2 & 16.4 & 11.4  & 57.2 & 22.1 & 17.6  & 31.1 & 8.9 & 2.2  & 12.4 & 4.6 & 3.1 \\
PHP & 98.8 & 89.0 & 73.3  & 52.9 & 34.0 & 31.0  & 45.1 & 37.5 & 35.8  & 39.4 & 33.5 & 31.3  & 23.5 & 22.4 & 20.9 \\
Perl & 100.0 & 89.3 & 67.8  & 59.1 & 29.0 & 19.5  & 47.0 & 30.1 & 20.0  & 16.9 & 12.2 & 5.4  & 18.9 & 13.7 & 9.3 \\
Python & 100.0 & 77.7 & 62.2  & 45.0 & 16.8 & 5.4  & 46.9 & 23.2 & 13.7  & 17.7 & 5.4 & 1.0  & 20.6 & 9.8 & 4.6 \\
Ruby & 100.0 & 91.4 & 59.9  & 55.0 & 13.4 & 3.4  & 56.7 & 27.1 & 12.8  & 29.3 & 15.4 & 3.6  & 36.4 & 21.4 & 11.0 \\
\textbf{Mean} & \textbf{97.0} & \textbf{76.3} & \textbf{60.1}  & \textbf{31.1} & \textbf{14.1} & \textbf{7.8}  & \textbf{30.8} & \textbf{18.8} & \textbf{12.7}  & \textbf{15.7} & \textbf{9.7} & \textbf{4.6}  & \textbf{12.9} & \textbf{8.7} & \textbf{5.4} \\
\end{tabular}
\end{center}
\caption{Failure modes for each language, fuzz operation, and phase (successful compilations, runs, and wrong output).}
\label{tbl:aggregated-per-language}
\end{table*}

The results for each fuzz type, phase, and language are summarized in
Table~\ref{tbl:aggregated-per-language}.
Predictably, the off-by-one integer perturbation ({\em IntPert})
was the fuzz that
compilers found most difficult to catch and the one that resulted
in most cases of erroneous output.
The C\# compiler seems to behave better than the others in this area,
having a significantly lower number of successful compilations than the
rest.
However, this did not result in a similarly good performance rank
regarding lack of erroneous output.

Identifier substitution ({\em IdSub}) and similar token substitution
({\em SimSub}) resulted in almost equal numbers of compilation failures.
Although one might expect that {\em IdSub} would be more difficult
to detect than the wider-ranging {\em SimSub},
our fuzzer's treatment of reserved words as identifiers
probably gave the game away by having {\em SimSub} introduce many
trivial syntax errors.
Nevertheless, {\em SimSub} resulted in a significantly higher number
of successful runs and consequent erroneous results.
Interestingly,
the erroneous results for {\em SimSub} were dramatically higher
than those for {\em IdSub} in the case of
languages with a more imaginative syntax, like Python, Haskell, and Ruby.

The random character substitution fuzz was the one that resulted in
the lowest number of erroneous results. However, it wreaked havoc with
{\sc php}'s compilation and output, and JavaScript's compilation. This
could be associated with how these languages treat source code with
non-{\sc ascii} characters. The random token substitution resulted in
the lowest number of successful compilations or runs and consequently
also a low number of erroneous results. However, {\sc php} performed
particularly badly in this area indicating a dangerously loose syntax.

To investigate the validity of our results, we carried out a
statistical analysis of the fuzz tests. In particular, we examined
whether the differences we see in the fuzzing results among languages
(e.g., compilation errors caught by the compiler) are statistically
significant. To do that we performed a $2\times 2$ contingency
analysis for all pairs of languages and each fuzz test. We used the
Fisher exact test \cite{Fis35},
instead of the more common chi square test,
since there were cases with very low frequencies.

The results are presented in Tables \ref{tbl:contingency-test-results-1} 
and \ref{tbl:contingency-test-results-2}. 
Black lozenges appear when a
particular fuzz test \emph{failed} to show a statistically significant
difference between two languages (significance was set at the 0.05
level). In each table, the results of the Fisher exact test for each
language vs.\ the other languages are presented.
Each row is divided into five bands, one for each fuzzing operation,
and for each band the tests are, in order, {\em compilation},
{\em execution (run)}, {\em output validity}.
The following results stand out.
\begin{itemize}
\item
The different results in fuzz tests between statically compiled and
dynamic languages are to a large extent statistically significant.
This validates the finding in Figure ~\ref{fig:results}
that less errors escape detection in static languages than dynamic.

\item C\# behaves more like C and C++ and less like Java, despite its
  surface similarities to the latter.

\item Haskell behaves more similarly to Java than other languages.

\item There are clusters of black longenes
(indicating a failure to show a significant difference)
between statically checked languages:
C and C++,
C++ and Java,
Haskell and Java.
However,
  we do not see a comparable pattern in dynamic languages.
  To paraphrase Tolstoy, it would seem that they are different in
  their own ways.
\end{itemize}

\begin{table*}
\centering
\subtable[C]{
\begin{tabular}{l c c c c c}
& {\scriptsize IdSub}&{\scriptsize IntPer}&{\scriptsize CharSub}&{\scriptsize TokSub}&{\scriptsize SimSub}\\
\hline

C++ & $\scriptscriptstyle\lozenge\blacklozenge\lozenge$&$\scriptscriptstyle\lozenge\lozenge\lozenge$&$\scriptscriptstyle\lozenge\blacklozenge\blacklozenge$&$\scriptscriptstyle\lozenge\blacklozenge\blacklozenge$&$\scriptscriptstyle\lozenge\blacklozenge\lozenge$ \\
C\# & $\scriptscriptstyle\lozenge\lozenge\lozenge$&$\scriptscriptstyle\lozenge\lozenge\lozenge$&$\scriptscriptstyle\lozenge\blacklozenge\lozenge$&$\scriptscriptstyle\lozenge\lozenge\lozenge$&$\scriptscriptstyle\lozenge\lozenge\lozenge$ \\
Haskell & $\scriptscriptstyle\lozenge\lozenge\lozenge$&$\scriptscriptstyle\lozenge\lozenge\lozenge$&$\scriptscriptstyle\lozenge\lozenge\lozenge$&$\scriptscriptstyle\lozenge\blacklozenge\lozenge$&$\scriptscriptstyle\lozenge\lozenge\lozenge$ \\
Java & $\scriptscriptstyle\lozenge\lozenge\lozenge$&$\scriptscriptstyle\blacklozenge\lozenge\lozenge$&$\scriptscriptstyle\lozenge\lozenge\blacklozenge$&$\scriptscriptstyle\lozenge\blacklozenge\lozenge$&$\scriptscriptstyle\lozenge\lozenge\blacklozenge$ \\
Javascript & $\scriptscriptstyle\lozenge\lozenge\lozenge$&$\scriptscriptstyle\blacklozenge\lozenge\lozenge$&$\scriptscriptstyle\lozenge\lozenge\lozenge$&$\scriptscriptstyle\lozenge\lozenge\lozenge$&$\scriptscriptstyle\lozenge\lozenge\lozenge$ \\
{\sc php} & $\scriptscriptstyle\lozenge\lozenge\lozenge$&$\scriptscriptstyle\lozenge\lozenge\lozenge$&$\scriptscriptstyle\lozenge\lozenge\lozenge$&$\scriptscriptstyle\lozenge\lozenge\lozenge$&$\scriptscriptstyle\lozenge\lozenge\lozenge$ \\
Perl & $\scriptscriptstyle\lozenge\lozenge\lozenge$&$\scriptscriptstyle\blacklozenge\lozenge\lozenge$&$\scriptscriptstyle\lozenge\lozenge\lozenge$&$\scriptscriptstyle\lozenge\lozenge\lozenge$&$\scriptscriptstyle\lozenge\lozenge\lozenge$ \\
Python & $\scriptscriptstyle\lozenge\lozenge\lozenge$&$\scriptscriptstyle\blacklozenge\lozenge\lozenge$&$\scriptscriptstyle\lozenge\lozenge\lozenge$&$\scriptscriptstyle\lozenge\lozenge\lozenge$&$\scriptscriptstyle\lozenge\lozenge\lozenge$ \\
Ruby & $\scriptscriptstyle\lozenge\lozenge\blacklozenge$&$\scriptscriptstyle\blacklozenge\lozenge\blacklozenge$&$\scriptscriptstyle\lozenge\lozenge\lozenge$&$\scriptscriptstyle\lozenge\lozenge\lozenge$&$\scriptscriptstyle\lozenge\lozenge\lozenge$ \\

\hline
& \\
\end{tabular}
} 
\subtable[C++]{
\begin{tabular}{l c c c c c}
& {\scriptsize IdSub}&{\scriptsize IntPer}&{\scriptsize CharSub}&{\scriptsize TokSub}&{\scriptsize SimSub}\\
\hline

C & $\scriptscriptstyle\lozenge\blacklozenge\lozenge$&$\scriptscriptstyle\lozenge\lozenge\lozenge$&$\scriptscriptstyle\lozenge\blacklozenge\blacklozenge$&$\scriptscriptstyle\lozenge\blacklozenge\blacklozenge$&$\scriptscriptstyle\lozenge\blacklozenge\lozenge$ \\
C\# & $\scriptscriptstyle\lozenge\lozenge\lozenge$&$\scriptscriptstyle\lozenge\lozenge\lozenge$&$\scriptscriptstyle\lozenge\blacklozenge\blacklozenge$&$\scriptscriptstyle\lozenge\lozenge\lozenge$&$\scriptscriptstyle\lozenge\lozenge\lozenge$ \\
Haskell & $\scriptscriptstyle\blacklozenge\lozenge\lozenge$&$\scriptscriptstyle\lozenge\lozenge\blacklozenge$&$\scriptscriptstyle\lozenge\lozenge\lozenge$&$\scriptscriptstyle\lozenge\blacklozenge\lozenge$&$\scriptscriptstyle\lozenge\blacklozenge\lozenge$ \\
Java & $\scriptscriptstyle\blacklozenge\lozenge\lozenge$&$\scriptscriptstyle\lozenge\lozenge\lozenge$&$\scriptscriptstyle\lozenge\lozenge\lozenge$&$\scriptscriptstyle\blacklozenge\blacklozenge\blacklozenge$&$\scriptscriptstyle\blacklozenge\lozenge\blacklozenge$ \\
Javascript & $\scriptscriptstyle\lozenge\lozenge\lozenge$&$\scriptscriptstyle\lozenge\lozenge\lozenge$&$\scriptscriptstyle\lozenge\lozenge\lozenge$&$\scriptscriptstyle\lozenge\lozenge\lozenge$&$\scriptscriptstyle\lozenge\lozenge\lozenge$ \\
{\sc php} & $\scriptscriptstyle\lozenge\lozenge\lozenge$&$\scriptscriptstyle\lozenge\lozenge\lozenge$&$\scriptscriptstyle\lozenge\lozenge\lozenge$&$\scriptscriptstyle\lozenge\lozenge\lozenge$&$\scriptscriptstyle\lozenge\blacklozenge\lozenge$ \\
Perl & $\scriptscriptstyle\lozenge\lozenge\lozenge$&$\scriptscriptstyle\lozenge\lozenge\lozenge$&$\scriptscriptstyle\lozenge\lozenge\lozenge$&$\scriptscriptstyle\lozenge\lozenge\lozenge$&$\scriptscriptstyle\lozenge\lozenge\blacklozenge$ \\
Python & $\scriptscriptstyle\lozenge\lozenge\blacklozenge$&$\scriptscriptstyle\lozenge\lozenge\lozenge$&$\scriptscriptstyle\lozenge\lozenge\lozenge$&$\scriptscriptstyle\lozenge\lozenge\lozenge$&$\scriptscriptstyle\lozenge\lozenge\blacklozenge$ \\
Ruby & $\scriptscriptstyle\lozenge\lozenge\lozenge$&$\scriptscriptstyle\lozenge\lozenge\lozenge$&$\scriptscriptstyle\lozenge\lozenge\lozenge$&$\scriptscriptstyle\lozenge\lozenge\lozenge$&$\scriptscriptstyle\lozenge\lozenge\lozenge$ \\

\hline
& \\
\end{tabular}
} 
\subtable[C\#]{
\begin{tabular}{l c c c c c}
& {\scriptsize IdSub}&{\scriptsize IntPer}&{\scriptsize CharSub}&{\scriptsize TokSub}&{\scriptsize SimSub}\\
\hline

C & $\scriptscriptstyle\lozenge\lozenge\lozenge$&$\scriptscriptstyle\lozenge\lozenge\lozenge$&$\scriptscriptstyle\lozenge\blacklozenge\lozenge$&$\scriptscriptstyle\lozenge\lozenge\lozenge$&$\scriptscriptstyle\lozenge\lozenge\lozenge$ \\
C++ & $\scriptscriptstyle\lozenge\lozenge\lozenge$&$\scriptscriptstyle\lozenge\lozenge\lozenge$&$\scriptscriptstyle\lozenge\blacklozenge\blacklozenge$&$\scriptscriptstyle\lozenge\lozenge\lozenge$&$\scriptscriptstyle\lozenge\lozenge\lozenge$ \\
Haskell & $\scriptscriptstyle\lozenge\lozenge\lozenge$&$\scriptscriptstyle\blacklozenge\lozenge\lozenge$&$\scriptscriptstyle\lozenge\lozenge\lozenge$&$\scriptscriptstyle\blacklozenge\lozenge\lozenge$&$\scriptscriptstyle\lozenge\lozenge\lozenge$ \\
Java & $\scriptscriptstyle\lozenge\lozenge\lozenge$&$\scriptscriptstyle\lozenge\lozenge\lozenge$&$\scriptscriptstyle\lozenge\lozenge\lozenge$&$\scriptscriptstyle\lozenge\lozenge\lozenge$&$\scriptscriptstyle\lozenge\lozenge\lozenge$ \\
Javascript & $\scriptscriptstyle\lozenge\lozenge\lozenge$&$\scriptscriptstyle\lozenge\lozenge\lozenge$&$\scriptscriptstyle\lozenge\lozenge\lozenge$&$\scriptscriptstyle\lozenge\lozenge\lozenge$&$\scriptscriptstyle\lozenge\lozenge\lozenge$ \\
{\sc php} & $\scriptscriptstyle\lozenge\lozenge\lozenge$&$\scriptscriptstyle\lozenge\lozenge\blacklozenge$&$\scriptscriptstyle\lozenge\lozenge\lozenge$&$\scriptscriptstyle\lozenge\blacklozenge\lozenge$&$\scriptscriptstyle\lozenge\lozenge\lozenge$ \\
Perl & $\scriptscriptstyle\lozenge\lozenge\lozenge$&$\scriptscriptstyle\lozenge\lozenge\lozenge$&$\scriptscriptstyle\lozenge\lozenge\lozenge$&$\scriptscriptstyle\lozenge\lozenge\lozenge$&$\scriptscriptstyle\lozenge\lozenge\lozenge$ \\
Python & $\scriptscriptstyle\lozenge\lozenge\lozenge$&$\scriptscriptstyle\lozenge\lozenge\lozenge$&$\scriptscriptstyle\lozenge\lozenge\lozenge$&$\scriptscriptstyle\lozenge\lozenge\lozenge$&$\scriptscriptstyle\lozenge\lozenge\lozenge$ \\
Ruby & $\scriptscriptstyle\lozenge\lozenge\lozenge$&$\scriptscriptstyle\lozenge\lozenge\lozenge$&$\scriptscriptstyle\lozenge\lozenge\lozenge$&$\scriptscriptstyle\lozenge\lozenge\lozenge$&$\scriptscriptstyle\lozenge\lozenge\lozenge$ \\

\hline
& \\
\end{tabular}
} 
\subtable[Haskell]{
\begin{tabular}{l c c c c c}
& {\scriptsize IdSub}&{\scriptsize IntPer}&{\scriptsize CharSub}&{\scriptsize TokSub}&{\scriptsize SimSub}\\
\hline

C & $\scriptscriptstyle\lozenge\lozenge\lozenge$&$\scriptscriptstyle\lozenge\lozenge\lozenge$&$\scriptscriptstyle\lozenge\lozenge\lozenge$&$\scriptscriptstyle\lozenge\blacklozenge\lozenge$&$\scriptscriptstyle\lozenge\lozenge\lozenge$ \\
C++ & $\scriptscriptstyle\blacklozenge\lozenge\lozenge$&$\scriptscriptstyle\lozenge\lozenge\blacklozenge$&$\scriptscriptstyle\lozenge\lozenge\lozenge$&$\scriptscriptstyle\lozenge\blacklozenge\lozenge$&$\scriptscriptstyle\lozenge\blacklozenge\lozenge$ \\
C\# & $\scriptscriptstyle\lozenge\lozenge\lozenge$&$\scriptscriptstyle\blacklozenge\lozenge\lozenge$&$\scriptscriptstyle\lozenge\lozenge\lozenge$&$\scriptscriptstyle\blacklozenge\lozenge\lozenge$&$\scriptscriptstyle\lozenge\lozenge\lozenge$ \\
Java & $\scriptscriptstyle\blacklozenge\blacklozenge\lozenge$&$\scriptscriptstyle\lozenge\lozenge\lozenge$&$\scriptscriptstyle\blacklozenge\blacklozenge\blacklozenge$&$\scriptscriptstyle\lozenge\blacklozenge\lozenge$&$\scriptscriptstyle\lozenge\lozenge\lozenge$ \\
Javascript & $\scriptscriptstyle\lozenge\lozenge\blacklozenge$&$\scriptscriptstyle\lozenge\lozenge\lozenge$&$\scriptscriptstyle\lozenge\lozenge\blacklozenge$&$\scriptscriptstyle\lozenge\lozenge\blacklozenge$&$\scriptscriptstyle\lozenge\lozenge\blacklozenge$ \\
{\sc php} & $\scriptscriptstyle\lozenge\blacklozenge\lozenge$&$\scriptscriptstyle\lozenge\lozenge\lozenge$&$\scriptscriptstyle\lozenge\lozenge\lozenge$&$\scriptscriptstyle\lozenge\lozenge\lozenge$&$\scriptscriptstyle\lozenge\blacklozenge\lozenge$ \\
Perl & $\scriptscriptstyle\lozenge\lozenge\blacklozenge$&$\scriptscriptstyle\lozenge\lozenge\lozenge$&$\scriptscriptstyle\lozenge\lozenge\lozenge$&$\scriptscriptstyle\lozenge\lozenge\blacklozenge$&$\scriptscriptstyle\lozenge\lozenge\lozenge$ \\
Python & $\scriptscriptstyle\lozenge\lozenge\lozenge$&$\scriptscriptstyle\lozenge\lozenge\lozenge$&$\scriptscriptstyle\lozenge\lozenge\blacklozenge$&$\scriptscriptstyle\lozenge\lozenge\lozenge$&$\scriptscriptstyle\lozenge\lozenge\lozenge$ \\
Ruby & $\scriptscriptstyle\lozenge\lozenge\lozenge$&$\scriptscriptstyle\lozenge\lozenge\lozenge$&$\scriptscriptstyle\lozenge\lozenge\blacklozenge$&$\scriptscriptstyle\lozenge\lozenge\lozenge$&$\scriptscriptstyle\lozenge\lozenge\lozenge$ \\

\hline
& \\
\end{tabular}
} 
\subtable[Java]{
\begin{tabular}{l c c c c c}
& {\scriptsize IdSub}&{\scriptsize IntPer}&{\scriptsize CharSub}&{\scriptsize TokSub}&{\scriptsize SimSub}\\
\hline

C & $\scriptscriptstyle\lozenge\lozenge\lozenge$&$\scriptscriptstyle\blacklozenge\lozenge\lozenge$&$\scriptscriptstyle\lozenge\lozenge\blacklozenge$&$\scriptscriptstyle\lozenge\blacklozenge\lozenge$&$\scriptscriptstyle\lozenge\lozenge\blacklozenge$ \\
C++ & $\scriptscriptstyle\blacklozenge\lozenge\lozenge$&$\scriptscriptstyle\lozenge\lozenge\lozenge$&$\scriptscriptstyle\lozenge\lozenge\lozenge$&$\scriptscriptstyle\blacklozenge\blacklozenge\blacklozenge$&$\scriptscriptstyle\blacklozenge\lozenge\blacklozenge$ \\
C\# & $\scriptscriptstyle\lozenge\lozenge\lozenge$&$\scriptscriptstyle\lozenge\lozenge\lozenge$&$\scriptscriptstyle\lozenge\lozenge\lozenge$&$\scriptscriptstyle\lozenge\lozenge\lozenge$&$\scriptscriptstyle\lozenge\lozenge\lozenge$ \\
Haskell & $\scriptscriptstyle\blacklozenge\blacklozenge\lozenge$&$\scriptscriptstyle\lozenge\lozenge\lozenge$&$\scriptscriptstyle\blacklozenge\blacklozenge\blacklozenge$&$\scriptscriptstyle\lozenge\blacklozenge\lozenge$&$\scriptscriptstyle\lozenge\lozenge\lozenge$ \\
Javascript & $\scriptscriptstyle\lozenge\lozenge\lozenge$&$\scriptscriptstyle\blacklozenge\lozenge\lozenge$&$\scriptscriptstyle\lozenge\lozenge\lozenge$&$\scriptscriptstyle\lozenge\lozenge\lozenge$&$\scriptscriptstyle\lozenge\lozenge\lozenge$ \\
{\sc php} & $\scriptscriptstyle\lozenge\blacklozenge\lozenge$&$\scriptscriptstyle\lozenge\lozenge\lozenge$&$\scriptscriptstyle\lozenge\blacklozenge\lozenge$&$\scriptscriptstyle\lozenge\lozenge\lozenge$&$\scriptscriptstyle\lozenge\lozenge\lozenge$ \\
Perl & $\scriptscriptstyle\lozenge\lozenge\lozenge$&$\scriptscriptstyle\blacklozenge\lozenge\blacklozenge$&$\scriptscriptstyle\lozenge\lozenge\lozenge$&$\scriptscriptstyle\lozenge\lozenge\lozenge$&$\scriptscriptstyle\lozenge\lozenge\lozenge$ \\
Python & $\scriptscriptstyle\lozenge\lozenge\lozenge$&$\scriptscriptstyle\blacklozenge\lozenge\lozenge$&$\scriptscriptstyle\lozenge\lozenge\blacklozenge$&$\scriptscriptstyle\lozenge\lozenge\lozenge$&$\scriptscriptstyle\lozenge\lozenge\blacklozenge$ \\
Ruby & $\scriptscriptstyle\lozenge\lozenge\lozenge$&$\scriptscriptstyle\blacklozenge\lozenge\lozenge$&$\scriptscriptstyle\lozenge\lozenge\lozenge$&$\scriptscriptstyle\lozenge\lozenge\lozenge$&$\scriptscriptstyle\lozenge\lozenge\lozenge$ \\

\hline
& \\
\end{tabular}
} 
\subtable[Javascript]{
\begin{tabular}{l c c c c c}
& {\scriptsize IdSub}&{\scriptsize IntPer}&{\scriptsize CharSub}&{\scriptsize TokSub}&{\scriptsize SimSub}\\
\hline

C & $\scriptscriptstyle\lozenge\lozenge\lozenge$&$\scriptscriptstyle\blacklozenge\lozenge\lozenge$&$\scriptscriptstyle\lozenge\lozenge\lozenge$&$\scriptscriptstyle\lozenge\lozenge\lozenge$&$\scriptscriptstyle\lozenge\lozenge\lozenge$ \\
C++ & $\scriptscriptstyle\lozenge\lozenge\lozenge$&$\scriptscriptstyle\lozenge\lozenge\lozenge$&$\scriptscriptstyle\lozenge\lozenge\lozenge$&$\scriptscriptstyle\lozenge\lozenge\lozenge$&$\scriptscriptstyle\lozenge\lozenge\lozenge$ \\
C\# & $\scriptscriptstyle\lozenge\lozenge\lozenge$&$\scriptscriptstyle\lozenge\lozenge\lozenge$&$\scriptscriptstyle\lozenge\lozenge\lozenge$&$\scriptscriptstyle\lozenge\lozenge\lozenge$&$\scriptscriptstyle\lozenge\lozenge\lozenge$ \\
Haskell & $\scriptscriptstyle\lozenge\lozenge\blacklozenge$&$\scriptscriptstyle\lozenge\lozenge\lozenge$&$\scriptscriptstyle\lozenge\lozenge\blacklozenge$&$\scriptscriptstyle\lozenge\lozenge\blacklozenge$&$\scriptscriptstyle\lozenge\lozenge\blacklozenge$ \\
Java & $\scriptscriptstyle\lozenge\lozenge\lozenge$&$\scriptscriptstyle\blacklozenge\lozenge\lozenge$&$\scriptscriptstyle\lozenge\lozenge\lozenge$&$\scriptscriptstyle\lozenge\lozenge\lozenge$&$\scriptscriptstyle\lozenge\lozenge\lozenge$ \\
{\sc php} & $\scriptscriptstyle\lozenge\lozenge\lozenge$&$\scriptscriptstyle\lozenge\lozenge\lozenge$&$\scriptscriptstyle\lozenge\lozenge\lozenge$&$\scriptscriptstyle\lozenge\lozenge\lozenge$&$\scriptscriptstyle\lozenge\lozenge\lozenge$ \\
Perl & $\scriptscriptstyle\lozenge\lozenge\blacklozenge$&$\scriptscriptstyle\blacklozenge\lozenge\lozenge$&$\scriptscriptstyle\lozenge\lozenge\lozenge$&$\scriptscriptstyle\lozenge\lozenge\blacklozenge$&$\scriptscriptstyle\lozenge\lozenge\lozenge$ \\
Python & $\scriptscriptstyle\lozenge\lozenge\lozenge$&$\scriptscriptstyle\blacklozenge\lozenge\lozenge$&$\scriptscriptstyle\lozenge\blacklozenge\lozenge$&$\scriptscriptstyle\lozenge\lozenge\lozenge$&$\scriptscriptstyle\lozenge\lozenge\lozenge$ \\
Ruby & $\scriptscriptstyle\lozenge\blacklozenge\lozenge$&$\scriptscriptstyle\blacklozenge\lozenge\lozenge$&$\scriptscriptstyle\lozenge\lozenge\blacklozenge$&$\scriptscriptstyle\lozenge\lozenge\lozenge$&$\scriptscriptstyle\blacklozenge\lozenge\lozenge$ \\

\hline
& \\
\end{tabular}
} 
\subtable[{\sc php}]{
\begin{tabular}{l c c c c c}
& {\scriptsize IdSub}&{\scriptsize IntPer}&{\scriptsize CharSub}&{\scriptsize TokSub}&{\scriptsize SimSub}\\
\hline

C & $\scriptscriptstyle\lozenge\lozenge\lozenge$&$\scriptscriptstyle\lozenge\lozenge\lozenge$&$\scriptscriptstyle\lozenge\lozenge\lozenge$&$\scriptscriptstyle\lozenge\lozenge\lozenge$&$\scriptscriptstyle\lozenge\lozenge\lozenge$ \\
C++ & $\scriptscriptstyle\lozenge\lozenge\lozenge$&$\scriptscriptstyle\lozenge\lozenge\lozenge$&$\scriptscriptstyle\lozenge\lozenge\lozenge$&$\scriptscriptstyle\lozenge\lozenge\lozenge$&$\scriptscriptstyle\lozenge\blacklozenge\lozenge$ \\
C\# & $\scriptscriptstyle\lozenge\lozenge\lozenge$&$\scriptscriptstyle\lozenge\lozenge\blacklozenge$&$\scriptscriptstyle\lozenge\lozenge\lozenge$&$\scriptscriptstyle\lozenge\blacklozenge\lozenge$&$\scriptscriptstyle\lozenge\lozenge\lozenge$ \\
Haskell & $\scriptscriptstyle\lozenge\blacklozenge\lozenge$&$\scriptscriptstyle\lozenge\lozenge\lozenge$&$\scriptscriptstyle\lozenge\lozenge\lozenge$&$\scriptscriptstyle\lozenge\lozenge\lozenge$&$\scriptscriptstyle\lozenge\blacklozenge\lozenge$ \\
Java & $\scriptscriptstyle\lozenge\blacklozenge\lozenge$&$\scriptscriptstyle\lozenge\lozenge\lozenge$&$\scriptscriptstyle\lozenge\blacklozenge\lozenge$&$\scriptscriptstyle\lozenge\lozenge\lozenge$&$\scriptscriptstyle\lozenge\lozenge\lozenge$ \\
Javascript & $\scriptscriptstyle\lozenge\lozenge\lozenge$&$\scriptscriptstyle\lozenge\lozenge\lozenge$&$\scriptscriptstyle\lozenge\lozenge\lozenge$&$\scriptscriptstyle\lozenge\lozenge\lozenge$&$\scriptscriptstyle\lozenge\lozenge\lozenge$ \\
Perl & $\scriptscriptstyle\lozenge\lozenge\lozenge$&$\scriptscriptstyle\lozenge\blacklozenge\lozenge$&$\scriptscriptstyle\lozenge\lozenge\lozenge$&$\scriptscriptstyle\lozenge\lozenge\lozenge$&$\scriptscriptstyle\blacklozenge\lozenge\lozenge$ \\
Python & $\scriptscriptstyle\lozenge\lozenge\lozenge$&$\scriptscriptstyle\lozenge\lozenge\lozenge$&$\scriptscriptstyle\lozenge\lozenge\lozenge$&$\scriptscriptstyle\lozenge\lozenge\lozenge$&$\scriptscriptstyle\blacklozenge\lozenge\lozenge$ \\
Ruby & $\scriptscriptstyle\lozenge\lozenge\lozenge$&$\scriptscriptstyle\lozenge\lozenge\lozenge$&$\scriptscriptstyle\lozenge\lozenge\lozenge$&$\scriptscriptstyle\lozenge\lozenge\lozenge$&$\scriptscriptstyle\lozenge\lozenge\lozenge$ \\

\hline
& \\
\end{tabular}
} 
\subtable[Perl]{
\begin{tabular}{l c c c c c}
& {\scriptsize IdSub}&{\scriptsize IntPer}&{\scriptsize CharSub}&{\scriptsize TokSub}&{\scriptsize SimSub}\\
\hline

C & $\scriptscriptstyle\lozenge\lozenge\lozenge$&$\scriptscriptstyle\blacklozenge\lozenge\lozenge$&$\scriptscriptstyle\lozenge\lozenge\lozenge$&$\scriptscriptstyle\lozenge\lozenge\lozenge$&$\scriptscriptstyle\lozenge\lozenge\lozenge$ \\
C++ & $\scriptscriptstyle\lozenge\lozenge\lozenge$&$\scriptscriptstyle\lozenge\lozenge\lozenge$&$\scriptscriptstyle\lozenge\lozenge\lozenge$&$\scriptscriptstyle\lozenge\lozenge\lozenge$&$\scriptscriptstyle\lozenge\lozenge\blacklozenge$ \\
C\# & $\scriptscriptstyle\lozenge\lozenge\lozenge$&$\scriptscriptstyle\lozenge\lozenge\lozenge$&$\scriptscriptstyle\lozenge\lozenge\lozenge$&$\scriptscriptstyle\lozenge\lozenge\lozenge$&$\scriptscriptstyle\lozenge\lozenge\lozenge$ \\
Haskell & $\scriptscriptstyle\lozenge\lozenge\blacklozenge$&$\scriptscriptstyle\lozenge\lozenge\lozenge$&$\scriptscriptstyle\lozenge\lozenge\lozenge$&$\scriptscriptstyle\lozenge\lozenge\blacklozenge$&$\scriptscriptstyle\lozenge\lozenge\lozenge$ \\
Java & $\scriptscriptstyle\lozenge\lozenge\lozenge$&$\scriptscriptstyle\blacklozenge\lozenge\blacklozenge$&$\scriptscriptstyle\lozenge\lozenge\lozenge$&$\scriptscriptstyle\lozenge\lozenge\lozenge$&$\scriptscriptstyle\lozenge\lozenge\lozenge$ \\
Javascript & $\scriptscriptstyle\lozenge\lozenge\blacklozenge$&$\scriptscriptstyle\blacklozenge\lozenge\lozenge$&$\scriptscriptstyle\lozenge\lozenge\lozenge$&$\scriptscriptstyle\lozenge\lozenge\blacklozenge$&$\scriptscriptstyle\lozenge\lozenge\lozenge$ \\
{\sc php} & $\scriptscriptstyle\lozenge\lozenge\lozenge$&$\scriptscriptstyle\lozenge\blacklozenge\lozenge$&$\scriptscriptstyle\lozenge\lozenge\lozenge$&$\scriptscriptstyle\lozenge\lozenge\lozenge$&$\scriptscriptstyle\blacklozenge\lozenge\lozenge$ \\
Python & $\scriptscriptstyle\lozenge\lozenge\lozenge$&$\scriptscriptstyle\blacklozenge\lozenge\lozenge$&$\scriptscriptstyle\blacklozenge\lozenge\lozenge$&$\scriptscriptstyle\lozenge\lozenge\lozenge$&$\scriptscriptstyle\blacklozenge\lozenge\lozenge$ \\
Ruby & $\scriptscriptstyle\lozenge\lozenge\lozenge$&$\scriptscriptstyle\blacklozenge\lozenge\lozenge$&$\scriptscriptstyle\lozenge\lozenge\lozenge$&$\scriptscriptstyle\lozenge\lozenge\lozenge$&$\scriptscriptstyle\lozenge\lozenge\lozenge$ \\

\hline
& \\
\end{tabular}
} 
\caption{Contingency test results for C, C++, C\#, Haskell, Java, Javascript, PHP, and Perl}
\label{tbl:contingency-test-results-1}
\end{table*}

\begin{table*}
\centering
\subtable[Python]{
\begin{tabular}{l c c c c c}
& {\scriptsize IdSub}&{\scriptsize IntPer}&{\scriptsize CharSub}&{\scriptsize TokSub}&{\scriptsize SimSub}\\
\hline

C & $\scriptscriptstyle\lozenge\lozenge\lozenge$&$\scriptscriptstyle\blacklozenge\lozenge\lozenge$&$\scriptscriptstyle\lozenge\lozenge\lozenge$&$\scriptscriptstyle\lozenge\lozenge\lozenge$&$\scriptscriptstyle\lozenge\lozenge\lozenge$ \\
C++ & $\scriptscriptstyle\lozenge\lozenge\blacklozenge$&$\scriptscriptstyle\lozenge\lozenge\lozenge$&$\scriptscriptstyle\lozenge\lozenge\lozenge$&$\scriptscriptstyle\lozenge\lozenge\lozenge$&$\scriptscriptstyle\lozenge\lozenge\blacklozenge$ \\
C\# & $\scriptscriptstyle\lozenge\lozenge\lozenge$&$\scriptscriptstyle\lozenge\lozenge\lozenge$&$\scriptscriptstyle\lozenge\lozenge\lozenge$&$\scriptscriptstyle\lozenge\lozenge\lozenge$&$\scriptscriptstyle\lozenge\lozenge\lozenge$ \\
Haskell & $\scriptscriptstyle\lozenge\lozenge\lozenge$&$\scriptscriptstyle\lozenge\lozenge\lozenge$&$\scriptscriptstyle\lozenge\lozenge\blacklozenge$&$\scriptscriptstyle\lozenge\lozenge\lozenge$&$\scriptscriptstyle\lozenge\lozenge\lozenge$ \\
Java & $\scriptscriptstyle\lozenge\lozenge\lozenge$&$\scriptscriptstyle\blacklozenge\lozenge\lozenge$&$\scriptscriptstyle\lozenge\lozenge\blacklozenge$&$\scriptscriptstyle\lozenge\lozenge\lozenge$&$\scriptscriptstyle\lozenge\lozenge\blacklozenge$ \\
Javascript & $\scriptscriptstyle\lozenge\lozenge\lozenge$&$\scriptscriptstyle\blacklozenge\lozenge\lozenge$&$\scriptscriptstyle\lozenge\blacklozenge\lozenge$&$\scriptscriptstyle\lozenge\lozenge\lozenge$&$\scriptscriptstyle\lozenge\lozenge\lozenge$ \\
{\sc php} & $\scriptscriptstyle\lozenge\lozenge\lozenge$&$\scriptscriptstyle\lozenge\lozenge\lozenge$&$\scriptscriptstyle\lozenge\lozenge\lozenge$&$\scriptscriptstyle\lozenge\lozenge\lozenge$&$\scriptscriptstyle\blacklozenge\lozenge\lozenge$ \\
Perl & $\scriptscriptstyle\lozenge\lozenge\lozenge$&$\scriptscriptstyle\blacklozenge\lozenge\lozenge$&$\scriptscriptstyle\blacklozenge\lozenge\lozenge$&$\scriptscriptstyle\lozenge\lozenge\lozenge$&$\scriptscriptstyle\blacklozenge\lozenge\lozenge$ \\
Ruby & $\scriptscriptstyle\lozenge\lozenge\lozenge$&$\scriptscriptstyle\blacklozenge\lozenge\lozenge$&$\scriptscriptstyle\lozenge\lozenge\lozenge$&$\scriptscriptstyle\lozenge\lozenge\blacklozenge$&$\scriptscriptstyle\lozenge\blacklozenge\lozenge$ \\

\hline
& \\
\end{tabular}
} 
\subtable[Ruby]{
\begin{tabular}{l c c c c c}
& {\scriptsize IdSub}&{\scriptsize IntPer}&{\scriptsize CharSub}&{\scriptsize TokSub}&{\scriptsize SimSub}\\
\hline

C & $\scriptscriptstyle\lozenge\lozenge\blacklozenge$&$\scriptscriptstyle\blacklozenge\lozenge\blacklozenge$&$\scriptscriptstyle\lozenge\lozenge\lozenge$&$\scriptscriptstyle\lozenge\lozenge\lozenge$&$\scriptscriptstyle\lozenge\lozenge\lozenge$ \\
C++ & $\scriptscriptstyle\lozenge\lozenge\lozenge$&$\scriptscriptstyle\lozenge\lozenge\lozenge$&$\scriptscriptstyle\lozenge\lozenge\lozenge$&$\scriptscriptstyle\lozenge\lozenge\lozenge$&$\scriptscriptstyle\lozenge\lozenge\lozenge$ \\
C\# & $\scriptscriptstyle\lozenge\lozenge\lozenge$&$\scriptscriptstyle\lozenge\lozenge\lozenge$&$\scriptscriptstyle\lozenge\lozenge\lozenge$&$\scriptscriptstyle\lozenge\lozenge\lozenge$&$\scriptscriptstyle\lozenge\lozenge\lozenge$ \\
Haskell & $\scriptscriptstyle\lozenge\lozenge\lozenge$&$\scriptscriptstyle\lozenge\lozenge\lozenge$&$\scriptscriptstyle\lozenge\lozenge\blacklozenge$&$\scriptscriptstyle\lozenge\lozenge\lozenge$&$\scriptscriptstyle\lozenge\lozenge\lozenge$ \\
Java & $\scriptscriptstyle\lozenge\lozenge\lozenge$&$\scriptscriptstyle\blacklozenge\lozenge\lozenge$&$\scriptscriptstyle\lozenge\lozenge\lozenge$&$\scriptscriptstyle\lozenge\lozenge\lozenge$&$\scriptscriptstyle\lozenge\lozenge\lozenge$ \\
Javascript & $\scriptscriptstyle\lozenge\blacklozenge\lozenge$&$\scriptscriptstyle\blacklozenge\lozenge\lozenge$&$\scriptscriptstyle\lozenge\lozenge\blacklozenge$&$\scriptscriptstyle\lozenge\lozenge\lozenge$&$\scriptscriptstyle\blacklozenge\lozenge\lozenge$ \\
{\sc php} & $\scriptscriptstyle\lozenge\lozenge\lozenge$&$\scriptscriptstyle\lozenge\lozenge\lozenge$&$\scriptscriptstyle\lozenge\lozenge\lozenge$&$\scriptscriptstyle\lozenge\lozenge\lozenge$&$\scriptscriptstyle\lozenge\lozenge\lozenge$ \\
Perl & $\scriptscriptstyle\lozenge\lozenge\lozenge$&$\scriptscriptstyle\blacklozenge\lozenge\lozenge$&$\scriptscriptstyle\lozenge\lozenge\lozenge$&$\scriptscriptstyle\lozenge\lozenge\lozenge$&$\scriptscriptstyle\lozenge\lozenge\lozenge$ \\
Python & $\scriptscriptstyle\lozenge\lozenge\lozenge$&$\scriptscriptstyle\blacklozenge\lozenge\lozenge$&$\scriptscriptstyle\lozenge\lozenge\lozenge$&$\scriptscriptstyle\lozenge\lozenge\blacklozenge$&$\scriptscriptstyle\lozenge\blacklozenge\lozenge$ \\

\hline
& \\
\end{tabular}
}
\caption{Contingency test results for Python and Ruby}
\label{tbl:contingency-test-results-2}
\end{table*}

\section{Related Work} % {{{1
\label{sec:related}

Comparative language evaluation has a long and sometimes colorful history.
See for instance,
the comparison of {\sc pl/i} with Cobol, {\sc fortran} and Jovial in
terms of programmer productivity and programmer efficiency~\cite{RWSB68};
the qualitative and quantitative comparison of Algol 60,
{\sc fortran}, Pascal and Algol 68~\cite{BOJO80};
Kernighan's delightful description of
Pascal's design and syntax flaws~\cite{Ker81};
as well as the relatively more recent study where code written
C, C++, Java, Perl, Python, Rexx, and Tcl
is compared in terms of execution time, memory consumption,
program size, and programmer productivity~\cite{PREC00}.

Our work introduces fuzzing as a method for programming language
evaluation.
Fuzzing as a technique to investigate the reliability of software
was first proposed in an article by Miller and his colleagues~\cite{MFS90}.
There they described how they tested common Unix
utilities in various operating systems and architectures and discovered that
25--33\% of these were crashing under certain conditions.

Nowadays fuzzing techniques are used mainly to detect software security
vulnerabilities and improve software reliability \cite{TJC08,GODE07}.
Several tools and techniques \cite{WWGZ11} have been developed,
introducing concepts like \textit{directed fuzz testing} \cite{GLRI09}.

Our experiment aims to exhibit the fault tolerance of each language
and, in particular, the extend to which a language can use features such as
its type system to shield programmers from errors~\cite{LYU95,KOKR07}.
The random fuzzing we employed in our study can be improved by taking
into account specific properties of the object being studied.
{\em Grammar-based white box fuzzing} \cite{God08},
takes into account the input language's grammar to fuzz the input in
ways that are syntactically correct.
This results in a higher rate of successful fuzzing and the location
of deeper problems.
Another interesting approach is H-fuzzing \cite{ZWZH11}:
a heuristic method that examines the execution paths of the program
to achieve higher path coverage.

Fuzz testing approaches are based on the fact that it is practically
impossible to determine all execution paths and all program inputs that will
fully test the validity and the reliability of a program.
The analogy to our study is that it is impossible to come up with all
the ways in which a programmer can write an incorrect program that the
compiler or run time system could detect.
Random testing \cite{HAM06} has been touted as a solution that can partially
deal with the aforementioned problem.
However it is not widely adopted outside the academic fields \cite{GGBO07},
because the techniques it introduces are difficult to apply
in complex systems and achieve good code coverage only at a significant cost
\cite{RAWO06}.
Similarly, {\em mutation testing}~\cite{JIHA10}
introduces errors in computer programs and then checks their
output against valid results.

In the introduction we mentioned that complex refactorings can
result in errors similar to the ones we are investigating.
A study of such errors appears in reference~\cite{DDGM07}.
Refactoring bugs result in corrupted code,
which is very difficult to detect, especially in the case of dynamic
languages~\cite{FFM11,SCHA12}.
Recent studies indicate that type systems are tightly related with
code maintainability and error detection~\cite{STHA11,KHRT12}.

\section{Conclusions and Further Work} % {{{1
\label{sec:conclusions}

The work we described in this study cries to be extended
by applying it on a larger and more diverse corpus of programming tasks.
It would also be interesting to test a wider variety of languages.
Although Haskell performed broadly similarly to the other strongly-typed
languages in our set we would hope that other declarative languages
would exhibit more interesting characteristics.
The fuzz operations can be also extended
and be made more realistic~\cite{LFGC07},
perhaps by implementing a mixture based on data from actual programming
errors.
Ideally, we would want to construct fuzz scenarios by taking
into account empirical evidence collected from developers working
in real-life situations~\cite{HANE10}.

In this study we tallied the failure modes associated with each
language and fuzz operation and reported the aggregate results.
Manually analyzing and categorizing the failure modes by looking
at the actual compilation and run time errors
is likely to produce interesting insights,
as well as feedback that can drive the
construction of better fuzz operations.

We already mentioned in Section~\ref{sec:results} that the
large degree of variation we witnessed among the scripting
language results may be a result of those languages'
more experimental nature.
More interestingly, this variation also suggests that
comparative language fuzz testing of the type we performed
can also be used to objectively evaluate programming language
designs.

Probably the most significant outcome of our study is the
demonstration of the potential of comparative language fuzz testing
for evaluating programming language designs.
While this study only evaluated the sensitivity of program behavior to typos,
other higher-level types of fuzzing
that simulate more complex programmer errors are certainly possible.
This opens the door into two broad research directions.

The first research direction involves the comparative evaluation
of programming languages using objective criteria,
such as the response of code implementing the same functionality
in diverse languages to fuzzing.
This is made significantly easier through the publication of
tasks implemented in numerous programming languages on the
{\em Rosetta Code} site.
Our community should therefore expend effort to contribute to
the site's wiki, increasing the trustworthiness and diversity of
the provided examples.

The second research strand involves the systematic study of
language design by using methods from the fields of reliability engineering
and software testing.
Again, fuzzing is just one technique, others could be inspired from
established methods like
hazard analysis,
fault tree analysis, and
test coverage analysis.

\acks

We would like to thank Florents Tselai and Konstantinos Stroggylos
for significant help in the porting and implementation of the
{\em Rosetta Code} tasks in our environment,
the numerous contributors of {\em Rosetta Code} for
making their efforts available to the public,
and the paper's reviewers for their many insightful comments.

This research has been co-financed by
the European Union (European Social Fund --- {\sc esf})
and Greek national funds through the Operational Program
``Education and Lifelong Learning''
of the National Strategic Reference Framework ({\sc nsrf})
--- Research Funding Program:
Thalis ---
Athens University of Economics and Business ---
Software Engineering Research Platform.

\paragraph{Code Availability} The source code for
the implemented tasks,
the fuzzer,
the language-specific methods, and
the driver are maintained on GitHub, and
are publicly available as open source software on \url{https://github.com/bkarak/fuzzer-plateau-2012}.

% We recommend abbrvnat bibliography style.

\bibliographystyle{abbrvnat}
\begin{thebibliography}{35}
\providecommand{\natexlab}[1]{#1}
\providecommand{\url}[1]{\texttt{#1}}
\expandafter\ifx\csname urlstyle\endcsname\relax
  \providecommand{\doi}[1]{doi: #1}\else
  \providecommand{\doi}{doi: \begingroup \urlstyle{rm}\Url}\fi

\bibitem[Boom and de~Jong(1980)]{BOJO80}
H.~J. Boom and E.~de~Jong.
\newblock A critical comparison of several programming language
  implementations.
\newblock \emph{Software: Practice and Experience}, 10\penalty0 (6):\penalty0
  435--473, 1980.
\newblock \doi{10.1002/spe.4380100605}.

\bibitem[Brader(1989)]{Brad89}
M.~Brader.
\newblock Mariner {I} [once more].
\newblock \emph{The Risks Digest}, 9\penalty0 (54), Dec. 1989.
\newblock URL \url{http://catless.ncl.ac.uk/Risks/9.54.html#subj1.1}.
\newblock Current August 6th, 2012.

\bibitem[Daniel et~al.(2007)Daniel, Dig, Garcia, and Marinov]{DDGM07}
B.~Daniel, D.~Dig, K.~Garcia, and D.~Marinov.
\newblock Automated testing of refactoring engines.
\newblock In \emph{Proceedings of the the 6th Joint Meeting of the European
  Software Engineering Conference and the ACM SIGSOFT Symposium on the
  Foundations of Software Engineering}, ESEC-FSE '07, pages 185--194, New York,
  NY, USA, 2007. ACM.
\newblock \doi{10.1145/1287624.1287651}.

\bibitem[Endres(1975)]{End75}
A.~Endres.
\newblock An analysis of errors and their causes in system programs.
\newblock \emph{SIGPLAN Notices}, 10\penalty0 (6):\penalty0 327--336, Apr.
  1975.
\newblock \doi{10.1145/390016.808455}.
\newblock Proceedings of the International Conference on Reliable Software.

\bibitem[Feldthaus et~al.(2011)Feldthaus, Millstein, M{\o}ller, Sch\"{a}fer,
  and Tip]{FFM11}
A.~Feldthaus, T.~Millstein, A.~M{\o}ller, M.~Sch\"{a}fer, and F.~Tip.
\newblock Tool-supported refactoring for {JavaScript}.
\newblock In \emph{Proceedings of the 2011 ACM International Conference on
  Object Oriented Programming Systems Languages and Applications}, OOPSLA '11,
  pages 119--138, New York, NY, USA, 2011. ACM.
\newblock \doi{10.1145/2048066.2048078}.

\bibitem[Fisher(1935)]{Fis35}
R.~A. Fisher.
\newblock The logic of inductive inference.
\newblock \emph{Journal of the Royal Statistical Society Series A},
  98:\penalty0 39--54, 1935.

\bibitem[Fowler(2000)]{Fow00}
M.~Fowler.
\newblock \emph{Refactoring: Improving the Design of Existing Code}.
\newblock Addison-Wesley, Boston, MA, 2000.
\newblock With contributions by Kent Beck, John Brant, William Opdyke, and Don
  Roberts.

\bibitem[Ganesh et~al.(2009)Ganesh, Leek, and Rinard]{GLRI09}
V.~Ganesh, T.~Leek, and M.~Rinard.
\newblock Taint-based directed whitebox fuzzing.
\newblock In \emph{Proceedings of the 31st International Conference on Software
  Engineering}, ICSE '09, pages 474--484, Washington, DC, USA, 2009. IEEE
  Computer Society.
\newblock \doi{10.1109/ICSE.2009.5070546}.

\bibitem[Gerlich et~al.(2007)Gerlich, Gerlich, and Boll]{GGBO07}
R.~Gerlich, R.~Gerlich, and T.~Boll.
\newblock Random testing: from the classical approach to a global view and full
  test automation.
\newblock In \emph{Proceedings of the 2nd International Workshop on Random
  Testing: Co-located with the 22nd IEEE/ACM International Conference on
  Automated Software Engineering (ASE 2007)}, RT '07, pages 30--37, New York,
  NY, USA, 2007. ACM.
\newblock \doi{10.1145/1292414.1292424}.

\bibitem[Godefroid(2007)]{GODE07}
P.~Godefroid.
\newblock Random testing for security: blackbox vs. whitebox fuzzing.
\newblock In \emph{Proceedings of the 2nd International Workshop on Random
  Testing: Co-located with the 22nd IEEE/ACM International Conference on
  Automated Software Engineering (ASE 2007)}, RT '07, pages 1--1, New York, NY,
  USA, 2007. ACM.
\newblock \doi{10.1145/1292414.1292416}.

\bibitem[Godefroid et~al.(2008)Godefroid, Kiezun, and Levin]{God08}
P.~Godefroid, A.~Kiezun, and M.~Y. Levin.
\newblock Grammar-based whitebox fuzzing.
\newblock In \emph{Proceedings of the 2008 ACM SIGPLAN Conference on
  Programming Language Design and Implementation}, PLDI '08, pages 206--215,
  New York, NY, USA, 2008. ACM.
\newblock \doi{10.1145/1375581.1375607}.

\bibitem[Hamlet(2006)]{HAM06}
D.~Hamlet.
\newblock When only random testing will do.
\newblock In \emph{Proceedings of the 1st International Workshop on Random
  Testing}, RT '06, pages 1--9, New York, NY, USA, 2006. ACM.
\newblock \doi{10.1145/1145735.1145737}.

\bibitem[Hanenberg(2010)]{HANE10}
S.~Hanenberg.
\newblock Faith, hope, and love: an essay on software science's neglect of
  human factors.
\newblock In \emph{OOPSLA '10: Proceedings of the ACM International Conference
  on Object Oriented Programming Systems Languages and Applications}, New York,
  NY, USA, 2010. ACM.
\newblock ISBN 978-1-4503-0203-6.
\newblock \doi{http://doi.acm.org/10.1145/1869459.1869536}.

\bibitem[Hofstadter(1989)]{Hof89}
D.~Hofstadter.
\newblock \emph{Godel, {E}scher, {B}ach: An Eternal Golden Braid}, page 137.
\newblock Vintage Books, 1989.

\bibitem[Jia and Harman(2010)]{JIHA10}
Y.~Jia and M.~Harman.
\newblock An analysis and survey of the development of mutation testing.
\newblock \emph{IEEE Transactions on Software Engineering}, \penalty0 (99),
  2010.

\bibitem[Kernighan(1981)]{Ker81}
B.~W. Kernighan.
\newblock Why {P}ascal is not my favorite programming language.
\newblock Computer Science Technical Report 100, Bell Laboratories, Murray
  Hill, NJ, July 1981.

\bibitem[King(2011)]{Kin11}
R.~S. King.
\newblock The top 10 programming languages.
\newblock \emph{IEEE Spectrum}, 48\penalty0 (10):\penalty0 84, Oct. 2011.
\newblock \doi{10.1109/MSPEC.2011.6027266}.

\bibitem[Kleinschmager et~al.(2012)Kleinschmager, Hanenberg, Robbes, Tanter,
  and Stefik]{KHRT12}
S.~Kleinschmager, S.~Hanenberg, R.~Robbes, {\'E}.~Tanter, and A.~Stefik.
\newblock Do static type systems improve the maintainability of software
  systems? {A}n empirical study.
\newblock In \emph{Proceedings of the International Conference on Program
  Comprehension}, pages 153--162, 2012.
\newblock \doi{10.1109/ICPC.2012.6240483}.

\bibitem[Knuth(1964)]{Knu64}
D.~Knuth.
\newblock Man or boy?
\newblock \emph{Algol Bulletin}, 17:\penalty0 7, July 1964.
\newblock URL
  \url{http://archive.computerhistory.org/resources/text/algol/algol_bulletin/A17/P24.HTM}.
\newblock Current August 7th, 2012.

\bibitem[Knuth(1989)]{KNUTH89}
D.~E. Knuth.
\newblock The errors of {TeX}.
\newblock \emph{Software: Practice and Experience}, 19\penalty0 (7):\penalty0
  607--687, July 1989.

\bibitem[Koren and Krishna(2007)]{KOKR07}
I.~Koren and C.~M. Krishna.
\newblock \emph{Fault-Tolerant Systems}.
\newblock Morgan Kaufmann Publishers Inc., San Francisco, CA, USA, 2007.

\bibitem[Lerner et~al.(2007)Lerner, Flower, Grossman, and Chambers]{LFGC07}
B.~S. Lerner, M.~Flower, D.~Grossman, and C.~Chambers.
\newblock Searching for type-error messages.
\newblock \emph{SIGPLAN Notices}, 42\penalty0 (6):\penalty0 425--434, June
  2007.
\newblock \doi{10.1145/1273442.1250783}.
\newblock Proceedings of the 2007 ACM SIGPLAN Conference on Programming
  Language Design and Implementation.

\bibitem[Lyu(1995)]{LYU95}
M.~R. Lyu.
\newblock \emph{Software Fault Tolerance}.
\newblock John Wiley \& Sons, Inc., New York, NY, USA, 1995.

\bibitem[Miller et~al.(1990)Miller, Fredriksen, and So]{MFS90}
B.~P. Miller, L.~Fredriksen, and B.~So.
\newblock An empirical study of the reliability of {UNIX} utilities.
\newblock \emph{Communications of the ACM}, 33\penalty0 (12):\penalty0 32--44,
  Dec. 1990.

\bibitem[Neumann(1995)]{Neu95}
P.~G. Neumann.
\newblock \emph{Computer Related Risks}, chapter 2.2.2 Other Space-Program
  Problems; {DO} I=1.10 bug in {M}ercury Software, page~27.
\newblock Addison-Wesley, Reading, MA, 1995.

\bibitem[Ostrand and Weyuker(1984)]{OW84}
T.~J. Ostrand and E.~J. Weyuker.
\newblock Collecting and categorizing software error data in an industrial
  environment.
\newblock \emph{Journal of Systems and Software}, 4\penalty0 (4):\penalty0
  289--300, 1984.
\newblock \doi{10.1016/0164-1212(84)90028-1}.

\bibitem[Pierce(2002)]{Pie02}
B.~C. Pierce.
\newblock \emph{Types and Programming Languages}.
\newblock MIT Press, 2002.

\bibitem[Prechelt(2000)]{PREC00}
L.~Prechelt.
\newblock An empirical comparison of seven programming languages.
\newblock \emph{Computer}, 33\penalty0 (10):\penalty0 23--29, Oct. 2000.
\newblock \doi{10.1109/2.876288}.

\bibitem[Ramler and Wolfmaier(2006)]{RAWO06}
R.~Ramler and K.~Wolfmaier.
\newblock Economic perspectives in test automation: balancing automated and
  manual testing with opportunity cost.
\newblock In \emph{Proceedings of the 2006 International Workshop on Automation
  of Software Test}, AST '06, pages 85--91, New York, NY, USA, 2006. ACM.
\newblock \doi{10.1145/1138929.1138946}.

\bibitem[Rubey et~al.(1968)Rubey, Wick, Stoner, and Bentley]{RWSB68}
R.~J. Rubey, R.~C. Wick, W.~J. Stoner, and L.~Bentley.
\newblock Comparative evaluation of {PL/I}.
\newblock Technical report, Logicon Inc, April 1968.
\newblock URL
  \url{http://oai.dtic.mil/oai/oai?verb=getRecord&metadataPrefix=html&identifier=AD0669096}.

\bibitem[Sch\"{a}fer(2012)]{SCHA12}
M.~Sch\"{a}fer.
\newblock Refactoring tools for dynamic languages.
\newblock In \emph{Proceedings of the Fifth Workshop on Refactoring Tools}, WRT
  '12, pages 59--62, New York, NY, USA, 2012. ACM.
\newblock \doi{10.1145/2328876.2328885}.

\bibitem[Stuchlik and Hanenberg(2011)]{STHA11}
A.~Stuchlik and S.~Hanenberg.
\newblock Static vs. dynamic type systems: an empirical study about the
  relationship between type casts and development time.
\newblock In \emph{Proceedings of the 7th Symposium on Dynamic Languages}, DLS
  '11, pages 97--106, New York, NY, USA, 2011. ACM.
\newblock ISBN 978-1-4503-0939-4.
\newblock \doi{10.1145/2047849.2047861}.

\bibitem[Takanen et~al.(2008)Takanen, DeMott, and Miller]{TJC08}
A.~Takanen, J.~DeMott, and C.~Miller.
\newblock \emph{Fuzzing for Software Security Testing and Quality Assurance}.
\newblock Artech House, Inc., Norwood, MA, USA, 1 edition, 2008.

\bibitem[Wang et~al.(2011)Wang, Wei, Gu, and Zou]{WWGZ11}
T.~Wang, T.~Wei, G.~Gu, and W.~Zou.
\newblock Checksum-aware fuzzing combined with dynamic taint analysis and
  symbolic execution.
\newblock \emph{ACM Transactions of Information Systems Security}, 14\penalty0
  (2):\penalty0 15:1--15:28, Sept. 2011.
\newblock \doi{10.1145/2019599.2019600}.

\bibitem[Zhao et~al.(2011)Zhao, Wen, and Zhao]{ZWZH11}
J.~Zhao, Y.~Wen, and G.~Zhao.
\newblock H-fuzzing: a new heuristic method for fuzzing data generation.
\newblock In \emph{Proceedings of the 8th IFIP International Conference on
  Network and Parallel Computing}, NPC'11, pages 32--43, Berlin, Heidelberg,
  2011. Springer-Verlag.

\end{thebibliography}

\end{document}
